\documentclass{beamer}

% --- Thème ---
% Vous pouvez choisir parmi de nombreux thèmes. Exemples : Madrid, Berlin, Warsaw, Montpellier, etc.
% Essayez différents thèmes pour voir celui qui vous plaît le plus.
\usetheme{Madrid}
% Vous pouvez aussi choisir des thèmes de couleurs, par exemple : dolphin, orchid, whale, beaver
\usecolortheme{default}

% --- Packages utiles ---
\usepackage[utf8]{inputenc}
\usepackage[T1]{fontenc}
\usepackage[frenchb]{babel} % Changed 'french' to 'frenchb'
\usepackage{graphicx} % Pour inclure des images
\usepackage{amsmath} % Pour les maths
\usepackage{booktabs} % Pour de jolis tableaux
\usepackage{hyperref} % Pour les liens
\usepackage{xurl}     % Pour mieux couper les URLs

% --- Informations pour la page de titre ---
\title[VSM-Tools Mi-Parcours]{Développement d'une application VSM-Tools}
\subtitle{Présentation de Mi-Parcours}
\author{Nom Membre 1 \and Nom Membre 2 \and Nom Membre 3 \and Nom Membre 4}
\institute{EIGSI} % Nom de votre institution
\date{3 Mai 2025} % Date de la présentation

% --- Logo (optionnel, dépend du thème) ---
% Placez votre logo ici si le thème le supporte bien (Madrid le fait)
\logo{\includegraphics[height=1cm]{../images/logo_eigsi.png}} % Ajustez height si besoin

\begin{document}

% --- Page de Titre ---
\begin{frame}
  \titlepage % Affiche la page de titre configurée ci-dessus
\end{frame}

% --- Table des Matières (optionnel) ---
\begin{frame}{Plan de la présentation}
  \tableofcontents % Génère automatiquement la table des matières basée sur les sections/sous-sections
\end{frame}

% --- Section 1 : Introduction & Contexte ---
\section{Introduction et Contexte}

\begin{frame}{Contexte du projet}
  \frametitle{Pourquoi VSM-Tools ?} % Titre de la diapositive

  \begin{itemize}
    \item Importance de l'optimisation des flux de valeur (VSM).
    \item Limites des outils actuels (Excel, papier, logiciels complexes).
    \item Objectif : Créer un outil dédié, simple et puissant.
  \end{itemize}

  % Suggestion Visuelle : Pyramide d'automatisation ou schéma VSM générique
  % \begin{figure}
  %   \includegraphics[width=0.7\textwidth]{../images/automation_pyramide.png}
  %   \caption{Positionnement de VSM dans l'industrie.}
  % \end{figure}
\end{frame}

\begin{frame}{Objectifs du projet}
  \begin{itemize}
    \item Développer une application web/desktop pour la modélisation VSM.
    \item Permettre le calcul automatique d'indicateurs clés.
    \item Faciliter l'identification des goulots d'étranglement.
    \item Offrir une interface utilisateur intuitive.
    % Ajoutez d'autres objectifs spécifiques
  \end{itemize}
\end{frame}

% --- Section 2 : Travail Réalisé ---
\section{Travail Réalisé}

\begin{frame}{Architecture et Technologies}
  \frametitle{Choix Techniques}

  \begin{columns}[T] % Divise la diapo en colonnes
    \begin{column}{0.5\textwidth} % Première colonne
      \textbf{Frontend:}
      \begin{itemize}
        \item React / TypeScript
        \item Vite
        \item Tailwind CSS
        \item Zustand (ou autre gestionnaire d'état)
      \end{itemize}
      \textbf{Backend/Desktop:}
      \begin{itemize}
        \item Electron
        \item Node.js
      \end{itemize}
    \end{column}
    \begin{column}{0.5\textwidth} % Deuxième colonne
      \textbf{Visualisation:}
      \begin{itemize}
        \item Librairie de diagrammes (Ex: React Flow, GoJS, etc. - à préciser)
      \end{itemize}
       % Suggestion Visuelle : Diagramme d'architecture simplifié
       % \begin{figure}
       %   \includegraphics[width=\textwidth]{../images/architecture_diagram.png} % Mettez le bon chemin
       % \end{figure}
    \end{column}
  \end{columns}
\end{frame}

\begin{frame}{Fonctionnalités Implémentées}
  \begin{itemize}
    \item Création/Sauvegarde/Chargement de cartes VSM.
    \item Palette d'éléments VSM (Processus, Stock, Transport...).
    \item Éditeur graphique (glisser-déposer, connexion).
    \item Panneau de propriétés pour les éléments.
    % Listez les fonctionnalités clés déjà développées
  \end{itemize}

  % Suggestion Visuelle : Capture d'écran de l'interface principale
  % \begin{figure}
  %   \includegraphics[width=\textwidth]{../images/img_4.png} % Mettez une capture pertinente
  % \end{figure}
\end{frame}

% --- Section 3 : Démonstration (Optionnel) ---
% \section{Démonstration}
% \begin{frame}{Démonstration Live}
%   % Prévoyez une diapo pour annoncer la démo si vous en faites une
%   Points clés à montrer :
%   \begin{itemize}
%     \item Création d'un processus simple.
%     \item Ajout d'indicateurs.
%     \item Sauvegarde.
%   \end{itemize}
% \end{frame}

% --- Section 4 : Prochaines Étapes ---
\section{Prochaines Étapes}

\begin{frame}{Planification}
  \frametitle{Travail à venir}
  \begin{itemize}
    \item Implémentation du calcul des indicateurs (Lead Time, Takt Time...).
    \item Finalisation de l'interface utilisateur.
    \item Ajout de fonctionnalités d'analyse (détection goulots).
    \item Tests et débogage.
    \item Rédaction de la documentation finale.
  \end{itemize}

  % Suggestion Visuelle : Diagramme de Gantt ou liste des sprints restants
  % \begin{figure}
  %   \includegraphics[width=0.8\textwidth]{../images/sprints_premiere_annee.png} % Exemple
  %   \caption{Planification des Sprints}
  % \end{figure}
\end{frame}

% --- Section 5 : Conclusion ---
\section{Conclusion}

\begin{frame}{Conclusion et Perspectives}
  \begin{itemize}
    \item Rappel des objectifs atteints à mi-parcours.
    \item Confirmation de la faisabilité du projet.
    \item Potentiel de l'outil VSM-Tools.
    \item Questions ?
  \end{itemize}

  % Suggestion Visuelle : Business Model Canvas ou un VSM état futur simplifié
  % \begin{figure}
  %   \includegraphics[width=0.6\textwidth]{../images/business_model_canvas.png}
  % \end{figure}
\end{frame}

% --- Diapositive de fin ---
\begin{frame}
  \centering
  {\Huge Merci de votre attention}
  \vspace{1cm} % Changed \[1cm] to \vspace{1cm}
  {\large Questions ?}
\end{frame}

\end{document}
