\documentclass{beamer}

% --- Thème ---
% Vous pouvez choisir parmi de nombreux thèmes. Exemples : Madrid, Berlin, Warsaw, Montpellier, etc.
% Essayez différents thèmes pour voir celui qui vous plaît le plus.
% \usetheme{Madrid}
\usetheme{metropolis} % Thème moderne et épuré
% Vous pouvez aussi choisir des thèmes de couleurs, par exemple : dolphin, orchid, whale, beaver
% \usecolortheme{default} % Metropolis gère les couleurs différemment

% Supprimer les symboles de navigation (si présents)
\setbeamertemplate{navigation symbols}{}

% --- Packages utiles ---
\usepackage[utf8]{inputenc}
\usepackage[T1]{fontenc}
\usepackage[frenchb]{babel} % Changed 'french' to 'frenchb'
\usepackage{graphicx} % Pour inclure des images
\usepackage{amsmath} % Pour les maths
\usepackage{booktabs} % Pour de jolis tableaux
\usepackage{hyperref} % Pour les liens
\usepackage{xurl}     % Pour mieux couper les URLs

% --- Informations pour la page de titre ---
\title[VSM-Tools Mi-Parcours]{Développement d'une application VSM-Tools}
\subtitle{Présentation de Mi-Parcours}
\author{Nom Membre 1 \and Nom Membre 2 \and Nom Membre 3 \and Nom Membre 4}
\institute{EIGSI} % Nom de votre institution
\date{3 Mai 2025} % Date de la présentation

\begin{document}

% --- Page de Titre ---
\begin{frame}
  % Logo EIGSI (optionnel, peut être ajouté manuellement si nécessaire)
  % \node[anchor=south east] at (current page.south east){\includegraphics[height=1cm]{../images/logo_eigsi.png}};
  \titlepage % Affiche la page de titre configurée ci-dessus
\end{frame}

% --- Table des Matières ---
% Frame option=shrink permet de réduire la taille si nécessaire pour tenir sur une page
\begin{frame}[shrink=15]{Plan de la présentation}
  \tableofcontents % Génère automatiquement la table des matières
\end{frame}

% --- Section 1 : Introduction & Contexte VSM ---
\section{Introduction & Contexte VSM}

\begin{frame}{Qu'est-ce que le VSM ?}
  \frametitle{Le Value Stream Mapping : Visualiser pour Optimiser}
  \begin{itemize}
    \item Méthode visuelle issue du Lean Management.
    \item Objectif : Cartographier TOUS les flux (matière, information) du fournisseur au client.
    \item But : Identifier les étapes à valeur ajoutée et les gaspillages.
    \item Fournit une image claire du processus actuel.
  \end{itemize}
\end{frame}

\begin{frame}{Importance et Bénéficiaires du VSM}
  \frametitle{Pourquoi le VSM est Essentiel ?}
  \textbf{Importance :}
  \begin{itemize}
    \item Identifier et quantifier les gaspillages (Muda).
    \item Localiser les goulots d'étranglement.
    \item Réduire les délais (Lead Time).
    \item Améliorer productivité et coûts.
    \item Favoriser l'amélioration continue (Kaizen).
  \end{itemize}
  \vfill % Espace vertical
  \textbf{Bénéficiaires :} Ingénieurs Process/Méthodes, Responsables Prod/Logistique/Qualité, Lean Managers, Consultants, Direction.
\end{frame}

\begin{frame}{Le Défi des Méthodes Traditionnelles}
  \frametitle{Limites des Outils Classiques}
  \begin{itemize}
    \item \textbf{Papier / Post-it :} Collaboratif mais difficile à maintenir, partager, analyser quantitativement.
    \item \textbf{Tableurs (Excel) :} Calculs possibles mais manque de visualisation VSM standardisée.
    \item \textbf{Logiciels de dessin (Visio) :} Flexibles mais sans intelligence métier VSM (pas de calculs auto, liens logiques...).
  \end{itemize}
  \vfill
  \textit{Besoin d'un outil dédié, simple et puissant.}
\end{frame}

% --- Section 2 : VSM-Tools : Notre Solution ---
\section{VSM-Tools : Notre Solution}

\begin{frame}{VSM-Tools : Mission et Vision}
  \frametitle{Notre Réponse aux Défis du VSM}
  \textbf{Mission :}
  \begin{itemize}
    \item Rendre l'analyse VSM \textbf{accessible, rapide et actionnable}.
    \item Faciliter l'amélioration de la performance opérationnelle.
  \end{itemize}
  \vfill
  \textbf{Vision :}
  \begin{itemize}
    \item Devenir un outil de \textbf{référence pour une cartographie VSM intuitive et efficace}.
    \item Guider l'utilisateur dans l'analyse et l'identification des améliorations.
  \end{itemize}
\end{frame}

\begin{frame}{Proposition de Valeur}
  \frametitle{Les Bénéfices Concrets de VSM-Tools}
  \begin{itemize}
    \item<1-> \textbf{Gain de temps significatif :} Création/mise à jour accélérées, calculs indicateurs automatisés (Lead Time, \%VA...).
    \item<2-> \textbf{Analyse approfondie et aide à la décision :}
    \begin{itemize}
        \item Identification rapide des goulots et gaspillages.
        \item Simulation d'impact des changements ("what-if").
        \item Anticipation des problèmes (prédictif).
        \item Suggestions d'optimisation.
    \end{itemize}
    \item<3-> \textbf{Collaboration facilitée :} Partage aisé des cartes et analyses.
    \item<4-> \textbf{Décisions éclairées et robustes :} Basées sur l'état actuel, scénarios futurs et analyses prédictives.
  \end{itemize}
\end{frame}

\begin{frame}{Positionnement et Atouts}
  \frametitle{Comment VSM-Tools se Distingue ?}
  \textbf{Panorama Existant :} Outils de dessin (Visio), Modules VSM intégrés (ERP/PLM), Outils VSM spécialisés (Simcad).
  \vfill
  \textbf{Atouts Différenciants de VSM-Tools :}
  \begin{itemize}
    \item \textbf{Approche équilibrée :} Simplicité + Puissance d'analyse.
    \item \textbf{Calculs dynamiques intégrés :} Indicateurs clés (LT, Takt, \%VA...) et timeline auto.
    \item \textbf{Capacités d'analyse avancées :} Simulation "what-if", Analyse prédictive, Suggestions d'optimisation.
    \item \textbf{Connectivité aux données} (potentielle).
    \item \textbf{Modularité et adaptabilité}.
  \end{itemize}
\end{frame}

% --- Section 3 : Fonctionnement de VSM-Tools ---
\section{Fonctionnement de VSM-Tools}

\begin{frame}{Parcours Utilisateur Fluide}
  \frametitle{De la Modélisation à l'Action}
  \begin{enumerate}
    \item \textbf{Modélisation visuelle intuitive :} Glisser-déposer des symboles VSM standards, connexion des flux.
    \item \textbf{Enrichissement facile des données :} Panneau dédié pour saisir les métriques (temps, opérateurs, stocks...).
    \item \textbf{Analyse guidée et approfondie :}
    \begin{itemize}
        \item Calculs auto des indicateurs \& timeline.
        \item Mise en évidence goulots, déséquilibres.
        \item \textit{Éléments prédictifs et suggestions d'optimisation.}
    \end{itemize}
    \item \textbf{Conception et simulation état futur :} Duplication carte, modélisation améliorations, simulation "what-if" pour valider l'impact.
    \item \textbf{Communication et partage :} Exports (PDF, image) pour rapports et présentations.
  \end{enumerate}
  % Optionnel : Ajouter une image de l'interface si disponible
  % \begin{figure} \includegraphics[width=0.6\textwidth]{../images/img_4.png} \end{figure}
\end{frame}

% --- Section 4 : Positionnement & Synergies ---
\section{Positionnement & Synergies}

\begin{frame}{Place dans l'Écosystème Industriel}
  \frametitle{VSM-Tools et la Pyramide d'Automatisation}
  \begin{columns}[T]
    \begin{column}{0.6\textwidth}
      \textbf{Pyramide d'Automatisation :}
      \begin{itemize}
        \item Stratégique (ERP)
        \item Pilotage (MES)
        \item Supervision (SCADA)
        \item Contrôle (PLC)
        \item Terrain (Capteurs)
      \end{itemize}
      \vfill
      \textbf{Position de VSM-Tools :}
      \begin{itemize}
        \item \textbf{En complément}, outil d'analyse transverse.
        \item Utilise données ERP/MES.
        \item Fournit analyses pour influencer stratégie (ERP) et guider actions (MES).
        \item Focalisé sur l'\textbf{amélioration des processus}.
      \end{itemize}
    \end{column}
    \begin{column}{0.4\textwidth}
      \begin{figure}
        \includegraphics[width=\textwidth]{../images/automation_pyramide.png}
        \caption{Pyramide d'Automatisation}
      \end{figure}
    \end{column}
  \end{columns}
\end{frame}

\begin{frame}{Synergies Potentielles}
  \frametitle{Interactions avec les Systèmes Existants}
  \begin{itemize}
    \item \textbf{Alimentation en données :} Récupération de données depuis MES/ERP pour une VSM "état actuel" plus rapide et précise.
    \item \textbf{Orientation des actions :} Objectifs issus de VSM-Tools (simulations, prédictions) pour paramétrer/piloter MES/ERP.
    \item \textbf{Complémentarité :}
    \begin{itemize}
        \item VSM-Tools : Comprendre, analyser, concevoir (diagnostic, simulation, prédiction).
        \item MES/SCADA : Exécuter, contrôler.
        \item ERP : Gérer globalement.
    \end{itemize}
    \item \textit{Boucle vertueuse : Analyse $\rightarrow$ Action $\rightarrow$ Données $\rightarrow$ Analyse.}
  \end{itemize}
\end{frame}

% --- Section 5 : Management de Projet & Avancement ---
\section{Management de Projet & Avancement}

\begin{frame}{Équipe et Méthodologie}
  \frametitle{Notre Organisation}
    \textbf{Équipe :} 4 élèves-ingénieurs EIGSI
    \begin{itemize}
        \item 3 x Supply Chain \& Transport International (Besoin fonctionnel, pertinence métier)
        \item 1 x IA \& Bases de Données (Aspects techniques, IA/prédictif)
    \end{itemize}
    \vfill
    \textbf{Méthodologie :}
    \begin{itemize}
        \item Approche \textbf{Agile légère} (inspirée Scrum/Kanban).
        \item Itératif et incrémental.
        \item Sprints thématiques (Année 1 : Conception).
        \item Tableau de suivi (Trello), Réunions régulières, Git (GitHub).
    \end{itemize}
\end{frame}

\begin{frame}{Planification et État d'Avancement}
  \frametitle{Où en Sommes-Nous ?}
  \textbf{Planification Année 1 (Conception) :}
  \begin{itemize}
    \item Sprints 1-2 : Cadrage, Recherche VSM & Concurrents.
    \item Sprint 3 : Besoins Fonctionnels & Cas d'Usage.
    \item Sprint 4 : Exigences Non-Fonctionnelles & Wireframes.
    \item Sprint 5 : Architecture Système & Choix Technos.
    \item Sprint 6 : Mockups & Modèle de Données.
    \item Sprint 7 : Finalisation Documentation Conception.
    \item Sprints 8-9 : Préparation Rapport Mi-Parcours.
  \end{itemize}
  \vfill
  \textbf{État d'Avancement (Mai 2025) :}
  \begin{itemize}
    \item Phase de conception (Sprints 1-7) \textbf{terminée}.
    \item Dossier de conception détaillé produit (exigences, archi, UI/UX, modèle données).
    \item Prêts pour la phase de développement (Année 2).
  \end{itemize}
  % Optionnel : Ajouter image Gantt/Sprints
  % \begin{figure} \includegraphics[width=0.8\textwidth]{../images/sprints_premiere_annee.png} \end{figure}
\end{frame}

% --- Section 6 : Prochaines Étapes & Vision ---
\section{Prochaines Étapes & Vision}

\begin{frame}{Prochaines Étapes}
  \frametitle{Vers la Réalisation}
  \textbf{Finalisation Année 1 :}
  \begin{itemize}
    \item Finalisation et soumission du Rapport Mi-Parcours.
    \item Validation finale du dossier de conception.
  \end{itemize}
  \vfill
  \textbf{Préparation Année 2 (Développement) :}
  \begin{itemize}
    \item Confirmation choix technologiques.
    \item Préparation du backlog initial pour le développement.
  \end{itemize}
  \vfill
  \textbf{Objectifs Année 2 :}
  \begin{itemize}
    \item Développement itératif des fonctionnalités (éditeur, calculs, simulation, IA...).
    \item Tests et validation.
    \item Livraison d'une version fonctionnelle de VSM-Tools.
  \end{itemize}
\end{frame}

\begin{frame}{Vision Future de VSM-Tools}
  \frametitle{Opportunités d'Évolution}
  \begin{itemize}
    \item \textbf{Intégrations poussées :} Connecteurs standards vers ERP/MES pour import auto des données.
    \item \textbf{Collaboration renforcée :} Fonctionnalités temps réel, gestion de versions, partage/commentaires améliorés.
    \item \textbf{Accessibilité et déploiement flexible :} Offre Cloud / SaaS.
    \item \textbf{Capitalisation des connaissances :} Bibliothèque de modèles VSM sectoriels, base de bonnes pratiques Lean intégrée.
  \end{itemize}
\end{frame}

% --- Section 7 : Conclusion ---
\section{Conclusion}

\begin{frame}{Conclusion}
  \frametitle{Synthèse et Perspectives}
  \begin{itemize}
    \item VSM : Outil clé mais limité par les méthodes traditionnelles.
    \item VSM-Tools : Solution dédiée, intuitive et \textbf{analytiquement puissante}.
    \item Bénéfices : Gain de temps, analyse approfondie (simulation, prédiction), aide à la décision.
    \item Positionnement : Complémentaire aux ERP/MES, focalisé sur l'optimisation des flux.
    \item Projet : Phase de conception terminée, prêt pour le développement.
    \item Vision : Intégrations, collaboration, accessibilité, capitalisation.
  \end{itemize}
  \vfill
  \textit{VSM-Tools : Un levier pour accélérer l'excellence opérationnelle.}
\end{frame}

% --- Diapositive de fin ---
\begin{frame}
  \centering
  {\Huge Merci de votre attention}
  \vspace{1cm}
  {\large Questions ?}
\end{frame}

\end{document}
