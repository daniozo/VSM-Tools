\documentclass{beamer}

% --- Thème ---
% Vous pouvez choisir parmi de nombreux thèmes. Exemples : Madrid, Berlin, Warsaw, Montpellier, etc.
% Essayez différents thèmes pour voir celui qui vous plaît le plus.
% \usetheme{Madrid}
\usetheme{metropolis} % Thème moderne et épuré
% Vous pouvez aussi choisir des thèmes de couleurs, par exemple : dolphin, orchid, whale, beaver
% \usecolortheme{default} % Metropolis gère les couleurs différemment

% Supprimer les symboles de navigation (si présents)
\setbeamertemplate{navigation symbols}{}

% --- Packages utiles ---
\usepackage[utf8]{inputenc}
\usepackage[T1]{fontenc}
\usepackage[frenchb]{babel} % Changed 'french' to 'frenchb'
\usepackage{graphicx} % Pour inclure des images
\usepackage{amsmath} % Pour les maths
\usepackage{booktabs} % Pour de jolis tableaux
\usepackage{hyperref} % Pour les liens
\usepackage{xurl}     % Pour mieux couper les URLs

% --- Informations pour la page de titre ---
\title[VSM-Tools Mi-Parcours]{Développement d'une application VSM-Tools}
\subtitle{Présentation de Mi-Parcours}
\author{MPOKE Jonathan \\
        PAMBO PAGA Chris Jeredh \\
        ZIDA Eben-Ezer \\
        ZOUGOU TOVIGNON Comlan Daniel}
\institute{EIGSI} % Nom de votre institution
\date{3 Mai 2025} % Date de la présentation

\begin{document}

% --- Page de Titre ---
\begin{frame}
  % Logo EIGSI (optionnel, peut être ajouté manuellement si nécessaire)
  % \node[anchor=south east] at (current page.south east){\includegraphics[height=1cm]{../images/logo_eigsi.png}};
  \titlepage % Affiche la page de titre configurée ci-dessus
\end{frame}

% --- Table des Matières ---
% Frame option=shrink permet de réduire la taille si nécessaire pour tenir sur une page
\begin{frame}{Plan de la présentation}
  \tableofcontents % Génère automatiquement la table des matières
\end{frame}

% --- Section 1 : Introduction & Contexte VSM ---
\section{Introduction & Contexte VSM}

\begin{frame}{Qu'est-ce que le VSM ?}
  \frametitle{Le Value Stream Mapping : Visualiser pour Optimiser}
  \begin{itemize}
    \item Méthode visuelle (Lean Management).
    \item Cartographie des flux (matière, information) : Fournisseur $\rightarrow$ Client.
    \item Objectif : Identifier valeur ajoutée vs gaspillages.
    \item Image claire du processus actuel.
  \end{itemize}
\end{frame}

\begin{frame}{Importance et Bénéficiaires du VSM}
  \frametitle{Pourquoi le VSM est Essentiel ?}
  \textbf{Importance :}
  \begin{itemize}
    \item Identifier \& quantifier gaspillages (Muda).
    \item Localiser goulots d'étranglement.
    \item Réduire délais (Lead Time).
    \item Améliorer productivité \& coûts.
    \item Base pour l'amélioration continue (Kaizen).
  \end{itemize}
  \vfill % Espace vertical
  \textbf{Bénéficiaires :} Ingénieurs, Responsables (Prod, Log, Qualité), Lean Managers, Direction.
\end{frame}

\begin{frame}{Le Défi des Méthodes Traditionnelles}
  \frametitle{Limites des Outils Classiques}
  \begin{itemize}
    \item \textbf{Papier/Post-it :} Difficile à maintenir, partager, analyser.
    \item \textbf{Tableurs (Excel) :} Calculs possibles, mais visualisation limitée.
    \item \textbf{Logiciels dessin (Visio) :} Flexibles, mais sans intelligence VSM (pas de calculs auto).
  \end{itemize}
  \vfill
  \textit{Besoin d'un outil dédié, simple et puissant.}
\end{frame}

% --- Section 2 : VSM-Tools : Notre Solution ---
\section{VSM-Tools : Notre Solution}

\begin{frame}{VSM-Tools : Mission et Vision}
  \frametitle{Notre Réponse aux Défis du VSM}
  \textbf{Mission :}
  \begin{itemize}
    \item Analyse VSM \textbf{accessible, rapide, actionnable}.
    \item Faciliter l'amélioration de la performance.
  \end{itemize}
  \vfill
  \textbf{Vision :}
  \begin{itemize}
    \item Outil de \textbf{référence} pour une cartographie VSM intuitive.
    \item Guider l'utilisateur dans l'analyse et l'optimisation.
  \end{itemize}
\end{frame}

\begin{frame}{Proposition de Valeur}
  \frametitle{Les Bénéfices Concrets de VSM-Tools}
  \begin{itemize}
    \item<1-> \textbf{Gain de temps :} Création rapide, calculs auto (LT, \%VA...). % LT = Lead Time
    \item<2-> \textbf{Analyse approfondie :} Goulots, gaspillages, simulation \textit{what-if}, prédictif, suggestions.
    \item<3-> \textbf{Collaboration facilitée :} Partage aisé.
    \item<4-> \textbf{Décisions éclairées :} Basées sur données, scénarios, prédictions.
  \end{itemize}
\end{frame}

\begin{frame}{Positionnement et Atouts}
  \frametitle{Comment VSM-Tools se Distingue ?}
  \textbf{Panorama :} Outils dessin (Visio), Modules intégrés (ERP/PLM), Outils spécialisés (Simcad).
  \vfill
  \textbf{Atouts VSM-Tools :}
  \begin{itemize}
    \item \textbf{Équilibre :} Simplicité + Puissance d'analyse.
    \item \textbf{Calculs dynamiques intégrés} (LT, Takt, \%VA...). % LT = Lead Time
    \item \textbf{Analyse avancée :} Simulation, Prédictif, Suggestions.
    \item Connectivité données (potentielle).
    \item Modularité.
  \end{itemize}
\end{frame}

% --- Section 3 : Fonctionnement de VSM-Tools ---
\section{Fonctionnement de VSM-Tools}

\begin{frame}{Parcours Utilisateur Fluide}
  \frametitle{De la Modélisation à l'Action}
  \small % Reduce font size slightly for this slide
  \begin{enumerate}
    \item \textbf{Modélisation visuelle :} Glisser-déposer symboles VSM, connexion flux.
    \item \textbf{Enrichissement données :} Panneau dédié (temps, opérateurs, stocks...). 
    \item \textbf{Analyse guidée :} Calculs auto (indicateurs, timeline), mise en évidence goulots, \textit{prédictif \& suggestions}.
    \item \textbf{Conception état futur :} Duplication carte, modélisation améliorations, simulation \textit{what-if}.
    \item \textbf{Communication :} Exports (PDF, image).
  \end{enumerate}
  % Optionnel : Ajouter une image de l'interface si disponible
  % \begin{figure} \includegraphics[width=0.5\textwidth]{../images/img_4.png} \end{figure}
\end{frame}

% --- Section 4 : Positionnement & Synergies ---
\section{Positionnement & Synergies}

\begin{frame}{Place dans l'Écosystème Industriel}
  \frametitle{VSM-Tools et la Pyramide d'Automatisation}
  \begin{columns}[T]
    \begin{column}{0.6\textwidth}
      \textbf{Pyramide :}
      \begin{itemize}
        \item Stratégique (ERP)
        \item Pilotage (MES)
        \item Supervision (SCADA)
        \item Contrôle (PLC)
        \item Terrain (Capteurs)
      \end{itemize}
      \vfill
      \textbf{Position VSM-Tools :}
      \begin{itemize}
        \item \textbf{Outil d'analyse transverse}, en complément.
        \item Utilise données ERP/MES.
        \item Influence stratégie (ERP) \& guide actions (MES).
        \item Focus : \textbf{Amélioration processus}.
      \end{itemize}
    \end{column}
    \begin{column}{0.4\textwidth}
      \begin{figure}
        \includegraphics[width=\textwidth]{../images/automation_pyramide.png}
        % \caption{Pyramide d'Automatisation}
      \end{figure}
    \end{column}
  \end{columns}
\end{frame}

\begin{frame}{Synergies Potentielles}
  \frametitle{Interactions avec les Systèmes Existants}
  \begin{itemize}
    \item \textbf{Alimentation données :} Depuis MES/ERP $\rightarrow$ VSM actuel précis \& rapide.
    \item \textbf{Orientation actions :} Objectifs VSM (simulations, prédictions) $\rightarrow$ Paramétrage MES/ERP.
    \item \textbf{Complémentarité :}
    \begin{itemize}
        \item VSM-Tools : Comprendre, analyser, concevoir.
        \item MES/SCADA : Exécuter, contrôler.
        \item ERP : Gérer globalement.
    \end{itemize}
    \item \textit{Boucle : Analyse $\rightarrow$ Action $\rightarrow$ Données $\rightarrow$ Analyse.}
  \end{itemize}
\end{frame}

% --- Section 5 : Management de Projet & Avancement ---
\section{Management de Projet & Avancement}

\begin{frame}{Équipe et Méthodologie}
  \frametitle{Notre Organisation}
    \textbf{Équipe :} 4 élèves-ingénieurs EIGSI
    \small
    \begin{itemize}
        \item 3 x Supply Chain \& Transport Intl. (Besoin fonctionnel, métier)
        \item 1 x IA \& Bases de Données (Technique, IA/prédictif)
    \end{itemize}
    \normalsize
    \vfill
    \textbf{Méthodologie :}
    \begin{itemize}
        \item \textbf{Agile légère} (inspirée Scrum/Kanban).
        \item Itératif et incrémental.
        \item Sprints thématiques (Année 1 : Conception).
        \item Outils : Trello, Réunions, GitHub.
    \end{itemize}
\end{frame}

\begin{frame}{Planification et État d'Avancement}
  \frametitle{Où en Sommes-Nous ?}
  \textbf{Planification Année 1 (Conception) :}
  \small
  \begin{itemize}
    \item Sprints 1-2 : Cadrage, Recherche.
    \item Sprint 3 : Besoins Fonctionnels.
    \item Sprint 4 : Exigences Non-Fonctionnelles, Wireframes.
    \item Sprint 5 : Architecture, Choix Technos.
    \item Sprint 6 : Mockups, Modèle Données.
    \item Sprint 7 : Finalisation Doc Conception.
    \item Sprints 8-9 : Rapport Mi-Parcours.
  \end{itemize}
  \normalsize
  \vfill
  \textbf{État d'Avancement (Mai 2025) :}
  \begin{itemize}
    \item Phase conception (Sprints 1-7) \textbf{terminée}.
    \item Dossier de conception détaillé produit.
    \item Prêts pour développement (Année 2).
  \end{itemize}
  % Optionnel : Ajouter image Gantt/Sprints
  % \begin{figure} \includegraphics[width=0.7\textwidth]{../images/sprints_premiere_annee.png} \end{figure}
\end{frame}

% --- Section 6 : Prochaines Étapes & Vision ---
\section{Prochaines Étapes & Vision}

\begin{frame}{Prochaines Étapes}
  \frametitle{Vers la Réalisation}
  \textbf{Fin Année 1 :}
  \begin{itemize}
    \item Finalisation Rapport Mi-Parcours.
    \item Validation dossier conception.
  \end{itemize}
  \vfill
  \textbf{Prépa Année 2 (Développement) :}
  \begin{itemize}
    \item Confirmation choix technos.
    \item Préparation backlog initial.
  \end{itemize}
  \vfill
  \textbf{Objectifs Année 2 :}
  \begin{itemize}
    \item Développement itératif (éditeur, calculs, simu, IA...). % simu = simulation
    \item Tests et validation.
    \item Livraison version fonctionnelle.
  \end{itemize}
\end{frame}

\begin{frame}{Vision Future de VSM-Tools}
  \frametitle{Opportunités d'Évolution}
  \begin{itemize}
    \item \textbf{Intégrations poussées :} Connecteurs ERP/MES.
    \item \textbf{Collaboration renforcée :} Temps réel, versions, partage.
    \item \textbf{Accessibilité :} Offre Cloud / SaaS.
    \item \textbf{Capitalisation connaissances :} Bibliothèque modèles, bonnes pratiques.
  \end{itemize}
\end{frame}

% --- Section 7 : Conclusion ---
\section{Conclusion}

\begin{frame}{Conclusion}
  \frametitle{Synthèse et Perspectives}
  \begin{itemize}
    \item VSM : Outil clé, mais méthodes trad. limitées.
    \item VSM-Tools : Solution dédiée, intuitive, \textbf{analytiquement puissante}.
    \item Bénéfices : Temps, analyse (simu, prédictif), décision.
    \item Positionnement : Complémentaire ERP/MES, focus optimisation flux.
    \item Projet : Conception terminée, prêt pour dev.
    \item Vision : Intégrations, collaboration, SaaS, capitalisation.
  \end{itemize}
  \vfill
\end{frame}

% --- Diapositive de fin ---
\begin{frame}
  \centering
  {\Huge Merci de votre attention}
  \vspace{1cm}
  {\large Questions ?}
\end{frame}

\end{document}
