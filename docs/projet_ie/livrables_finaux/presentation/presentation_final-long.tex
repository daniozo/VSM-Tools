\documentclass{beamer}

% --- Thème ---
\usetheme{metropolis} % Thème moderne et épuré

% Supprimer les symboles de navigation
\setbeamertemplate{navigation symbols}{}

% --- Packages utiles ---
\usepackage[utf8]{inputenc}
\usepackage[T1]{fontenc}
\usepackage{lmodern}
\usepackage[french, provide=*]{babel} % Pour la typographie française
\usepackage{graphicx} % Pour inclure des images
\usepackage{amsmath} % Pour les maths
\usepackage{booktabs} % Pour de jolis tableaux
\usepackage{hyperref} % Pour les liens
\usepackage{xurl}     % Pour mieux couper les URLs
\usepackage{tikz}     % Pour positionner le logo

% --- Informations pour la page de titre ---
\title[VSM-Tools - Rapport Final]{Développement d'une application de visualisation des flux de production pour identifier les goulots d'étranglement et optimiser les performances industrielles}
\subtitle{Présentation Finale}
\author{MPOKE Jonathan \\
        PAMBO PAGA Chris Jeredh \\
        ZIDA Eben-Ezer \\
        ZOUGOU TOVIGNON Comlan Daniel}
\institute{EIGSI}
\date{Année académique : 2025 - 2026}

% --- Logo sur chaque slide (coin supérieur droit) ---
\addtobeamertemplate{frametitle}{}{%
  \begin{tikzpicture}[remember picture,overlay]
    \node[anchor=north east,yshift=2pt] at (current page.north east) {%
      \includegraphics[height=0.8cm]{../../../images/logo_eigsi.png}%
    };
  \end{tikzpicture}%
}

\setbeamertemplate{footline}{} % Supprime la ligne de pied de page

\begin{document}

% --- Page de Titre ---
\begin{frame}
  \centering
  \includegraphics[width=0.25\textwidth]{../../../images/logo_eigsi.png}
  \vspace{0.5cm}
  \titlepage
\end{frame}

% --- Table des Matières ---
\begin{frame}{Plan de la présentation}
  \tableofcontents
\end{frame}

% ==============================================================================
% SECTION 1 : INTRODUCTION
% ==============================================================================
\section{Introduction}

\begin{frame}{Contexte du Projet}
  \frametitle{Introduction : Le Défi de l'Industrie Moderne}
  \begin{itemize}
    \item L'industrie moderne : quête perpétuelle d'\textbf{efficacité}.
    \item Compétition mondiale + exigences clients croissantes.
    \item \textbf{Nécessité stratégique} : visualiser, comprendre et améliorer les processus.
  \end{itemize}
  \vfill
  \textbf{Notre réponse :} \\
  Une solution logicielle dédiée basée sur le \textbf{Value Stream Mapping (VSM)}.
\end{frame}

\begin{frame}{Objectif de la Présentation}
  \frametitle{Ce que nous allons vous présenter}
  \begin{enumerate}
    \item Les \textbf{enjeux} de l'optimisation industrielle.
    \item La méthodologie \textbf{VSM} comme outil de clarification.
    \item \textbf{VSM-Tools} : notre solution et son architecture.
    \item \textbf{Parcours utilisateur} et fonctionnalités clés.
    \item \textbf{Démonstration} de l'application.
  \end{enumerate}
\end{frame}

% ==============================================================================
% SECTION 2 : LES ENJEUX DE L'OPTIMISATION INDUSTRIELLE
% ==============================================================================
\section{Les enjeux de l'optimisation industrielle}

\begin{frame}{Un Contexte de Pression Continue}
  \frametitle{Les Défis des Entreprises Industrielles}
  \begin{itemize}
    \item \textbf{Concurrence mondiale :} Produire mieux, plus vite, moins cher.
    \item \textbf{Exigence des clients :} Haute qualité, personnalisation, délais courts.
    \item \textbf{Volatilité des marchés :} Chaînes d'approvisionnement complexes et fragiles.
    \item \textbf{Pression sur les ressources :} Coûts croissants, réglementations environnementales.
  \end{itemize}
  \vfill
  \textit{L'optimisation n'est plus un confort, c'est une condition de survie.}
\end{frame}

\begin{frame}{Les Piliers de la Performance}
  \frametitle{Le Triptyque QCD : Qualité - Coûts - Délais}
  \begin{itemize}
    \item \textbf{Maîtriser les Coûts :}
          \begin{itemize}
            \item Éliminer les gaspillages (Muda).
            \item Surproduction, stocks excessifs, temps d'attente, défauts\ldots
          \end{itemize}
    \item \textbf{Améliorer la Qualité :}
          \begin{itemize}
            \item Tendre vers le « zéro défaut ».
            \item Construire la qualité au cœur du processus.
          \end{itemize}
    \item \textbf{Réduire les Délais :}
          \begin{itemize}
            \item Réduire le \textit{Lead Time} (commande $\rightarrow$ livraison).
            \item Améliorer satisfaction client et flexibilité.
          \end{itemize}
  \end{itemize}
\end{frame}

\begin{frame}{Le Défi de la Visibilité}
  \frametitle{De la Donnée à la Décision}
  \textbf{Principal obstacle :} Le manque de \textbf{visibilité}.
  \vfill
  \begin{itemize}
    \item Processus = systèmes complexes avec dépendances cachées.
    \item Gaspillages et goulots souvent \textbf{noyés dans la complexité}.
    \item Efforts d'amélioration locaux, parcellaires, parfois contre-productifs.
  \end{itemize}
  \vfill
  \centering
  \textbf{\textit{Pour optimiser, il faut d'abord VOIR.}}
\end{frame}

% ==============================================================================
% SECTION 3 : LE VALUE STREAM MAPPING
% ==============================================================================
\section{Le Value Stream Mapping}

\begin{frame}{Qu'est-ce que le VSM ?}
  \frametitle{Value Stream Mapping : Définition}
  \begin{itemize}
    \item Méthode visuelle issue du \textbf{Lean Management}.
    \item Cartographie des \textbf{flux} (matière et information).
    \item Du \textbf{Fournisseur} jusqu'au \textbf{Client}.
    \item Objectif : Visualiser \textbf{toutes} les étapes (valeur ajoutée ou non).
  \end{itemize}
  \vfill
  \textit{L'analyse ne se limite pas aux machines, mais inclut ce qui se passe \textbf{entre} elles : temps d'attente, stocks, mouvements.}
\end{frame}

\begin{frame}{Du VSM État Actuel au VSM État Futur}
  \frametitle{Exemples de Value Stream Mapping}
  \begin{columns}[c]
    \begin{column}{0.45\textwidth}
      \centering
      \includegraphics[width=\textwidth]{../../../images/vsm_etat_actuel}
      \textbf{État Actuel}
    \end{column}

    \begin{column}{0.1\textwidth}
      \centering
      \LARGE{$\Rightarrow$}
    \end{column}

    \begin{column}{0.45\textwidth}
      \centering
      \includegraphics[width=\textwidth]{../../../images/vsm_etat_futur}
      \textbf{État Futur}
    \end{column}
  \end{columns}
\end{frame}

\begin{frame}{Importance du VSM}
  \frametitle{Pourquoi le VSM est Essentiel ?}
  \begin{itemize}
    \item \textbf{Identifier \& quantifier} les gaspillages (Muda).
    \item \textbf{Localiser} précisément les goulots d'étranglement.
    \item \textbf{Réduire} les délais de production (Lead Time).
    \item \textbf{Améliorer} la productivité et réduire les coûts.
    \item \textbf{Base} pour l'amélioration continue (Kaizen).
  \end{itemize}
  \vfill
  \textbf{Bénéficiaires :} Ingénieurs process, Responsables (Prod, Log, Qualité), Lean Managers, Direction.
\end{frame}

\begin{frame}{Limites des Méthodes Traditionnelles}
  \frametitle{Le Défi des Outils Classiques}
  \begin{itemize}
    \item \textbf{Papier/Post-it :}
          \begin{itemize}
            \item Difficile à maintenir, partager, analyser quantitativement.
          \end{itemize}
    \item \textbf{Tableurs (Excel) :}
          \begin{itemize}
            \item Calculs possibles, mais visualisation limitée et non standardisée.
          \end{itemize}
    \item \textbf{Logiciels de dessin (Visio, Draw.io) :}
          \begin{itemize}
            \item Flexibles, mais \textbf{sans intelligence métier}.
            \item Pas de calculs automatiques, pas de liens logiques.
            \item La carte reste un \textbf{dessin statique}.
          \end{itemize}
  \end{itemize}
  \vfill
  \centering
  \textit{Besoin d'un outil dédié, simple et puissant.}
\end{frame}

% ==============================================================================
% SECTION 4 : VSM-TOOLS : NOTRE SOLUTION
% ==============================================================================
\section{VSM-Tools : Notre Solution}

\begin{frame}{Mission et Vision}
  \frametitle{VSM-Tools : Notre Réponse aux Défis}
  \textbf{Mission :}
  \begin{itemize}
    \item Rendre l'analyse VSM \textbf{accessible, rapide et actionnable}.
    \item Lever les barrières liées à la complexité des outils ou à la collecte de données.
  \end{itemize}
  \vfill
  \textbf{Vision :}
  \begin{itemize}
    \item Devenir un outil de \textbf{référence} pour une cartographie VSM intuitive.
    \item \textbf{Guider} l'utilisateur dans l'analyse et l'identification des pistes d'amélioration.
  \end{itemize}
\end{frame}

\begin{frame}{Les 4 Apports Clés}
  \frametitle{Comment VSM-Tools se Distingue ?}
  \begin{enumerate}
    \item \textbf{Simplification du processus VSM}
          \begin{itemize}
            \item Modélisation guidée (approche « Model-First »).
            \item Calculs automatiques (Lead Time, \% Valeur Ajoutée).
          \end{itemize}
    \item \textbf{Dynamisation des données}
          \begin{itemize}
            \item 3 modes : Statique, Manuel (opérateur), Dynamique (SQL/API).
          \end{itemize}
    \item \textbf{Analyse augmentée par l'intelligence}
          \begin{itemize}
            \item Détection des goulots, suggestions d'optimisation.
            \item Assistant conversationnel (IA) intégré.
          \end{itemize}
    \item \textbf{Architecture moderne et évolutive}
          \begin{itemize}
            \item Studio (Electron) + Engine (Node.js) + Interfaces Web (React).
          \end{itemize}
  \end{enumerate}
\end{frame}

\begin{frame}{Gestion du Cycle VSM}
  \frametitle{État Actuel $\rightarrow$ État Futur}
  \begin{itemize}
    \item \textbf{Duplication facile} d'une cartographie « état actuel ».
    \item Modélisation des \textbf{améliorations} sur l'état futur.
    \item \textbf{Mise à jour instantanée} des indicateurs globaux.
    \item \textbf{Quantification de l'impact} des propositions avant déploiement.
  \end{itemize}
  \vfill
  \textbf{+ Gestion de projet intégrée :}
  \begin{itemize}
    \item Plan d'action, notes, import/export \texttt{.vsmx}.
    \item Base de données centralisée.
  \end{itemize}
\end{frame}

% ==============================================================================
% SECTION 5 : ARCHITECTURE TECHNIQUE
% ==============================================================================
\section{Architecture Technique}

\begin{frame}{Vue d'Ensemble de l'Architecture}
  \frametitle{Une Architecture Modulaire et Découplée}
  \begin{figure}
    \centering
    \includegraphics[width=0.75\textwidth]{../../../images/architecture_vsm_tools}
    \caption{Architecture générale de VSM-Tools}
  \end{figure}

  \textbf{3 composants principaux :}
  \begin{itemize}
    \item VSM Studio (Electron)
    \item VSM Engine (Node.js)
    \item Base de données PostgreSQL
  \end{itemize}
\end{frame}

\begin{frame}{Le VSM Studio}
  \frametitle{L'Application de Bureau (Electron)}
  \textbf{Pourquoi Electron ?}
  \begin{itemize}
    \item Application de bureau \textbf{multi-plateformes} (Windows, macOS, Linux).
    \item Technologies web standard (HTML, CSS, JavaScript).
  \end{itemize}
  \vfill
  \textbf{2 types de processus :}
  \begin{itemize}
    \item \textbf{Processus Principal :}
          \begin{itemize}
            \item Gestion des fenêtres, menus natifs.
            \item Communication avec l'Engine et la base de données.
          \end{itemize}
    \item \textbf{Processus de Rendu :}
          \begin{itemize}
            \item Interface utilisateur (navigateur Chromium).
            \item Affichage des diagrammes et formulaires.
          \end{itemize}
  \end{itemize}
\end{frame}

\begin{frame}{Le VSM Engine}
  \frametitle{Le Moteur de Services (Node.js)}
  \textbf{Service en arrière-plan} lancé par le Studio.
  \vfill
  \textbf{Responsabilités :}
  \begin{itemize}
    \item \textbf{Logique métier :} Validation du modèle VSM, calculs automatiques.
    \item \textbf{Gestion de la base de données :} Communication avec PostgreSQL.
    \item \textbf{Connectivité externe :} Connecteurs SQL et API REST pour le mode dynamique.
    \item \textbf{API REST interne :} Exposition des données pour le Studio et les interfaces web.
  \end{itemize}
\end{frame}

\begin{frame}{Base de Données Centralisée}
  \frametitle{PostgreSQL : Le Cœur du Système}
  \textbf{Avantages vs. stockage local :}
  \begin{itemize}
    \item \textbf{Centralisation et partage :} Plusieurs utilisateurs peuvent accéder aux mêmes projets.
    \item \textbf{Robustesse et intégrité :} Transactions ACID, contraintes d'intégrité.
    \item \textbf{Performance et scalabilité :} Gestion efficace de grands volumes de données.
    \item \textbf{Accès unifié :} L'Engine est la seule couche d'accès (sécurité).
  \end{itemize}
  \vfill
  \textit{VSM-Tools = Plateforme d'entreprise, pas un simple outil isolé.}
\end{frame}

\begin{frame}{Les Interfaces Web}
  \frametitle{Interfaces React pour Tous les Acteurs}
  \textbf{2 applications web légères :}
  \begin{enumerate}
    \item \textbf{Interface de saisie opérateur}
          \begin{itemize}
            \item Ultra-simplifiée, accessible via navigateur.
            \item Saisie des valeurs d'indicateurs depuis l'atelier.
          \end{itemize}
    \item \textbf{Dashboard de visualisation}
          \begin{itemize}
            \item Lecture seule pour le management.
            \item Consultation des diagrammes VSM à jour.
          \end{itemize}
  \end{enumerate}
  \vfill
  \textbf{Servies par l'Engine}, consomment la même API REST.
\end{frame}

\begin{frame}{Intégration de l'IA}
  \frametitle{Assistant Conversationnel Intégré}
  \textbf{Agent conversationnel} dans le VSM Studio :
  \begin{itemize}
    \item Alimenté par un \textbf{modèle de langage} (LLM).
    \item Communication gérée par le \textbf{Processus Principal} (sécurité).
    \item Capacités :
          \begin{itemize}
            \item Répondre aux questions sur la méthodologie VSM.
            \item Guider dans l'utilisation du logiciel.
            \item Interpréter les données et suggérer des améliorations.
          \end{itemize}
  \end{itemize}
  \vfill
  \textit{Clés d'API sécurisées, jamais exposées côté client.}
\end{frame}

% ==============================================================================
% SECTION 6 : PARCOURS UTILISATEUR
% ==============================================================================
\section{Parcours Utilisateur}

\begin{frame}{Trois Parcours, Trois Acteurs}
  \frametitle{Des Interfaces Adaptées à Chaque Rôle}
  \begin{enumerate}
    \item \textbf{L'Analyste} : Conception et analyse (VSM Studio).
    \item \textbf{L'Opérateur} : Collecte de données terrain (Interface Web).
    \item \textbf{Le Manager} : Supervision et décision (Dashboard Web).
  \end{enumerate}
  \vfill
  \textit{Tous s'articulent autour de la même base de données centrale.}
\end{frame}

\begin{frame}{Parcours de l'Analyste (1/4)}
  \frametitle{1. Initialisation du Projet}
  \begin{itemize}
    \item Création d'un nouveau projet dans la base de données.
    \item Ou import d'un fichier \texttt{.vsmx} existant.
  \end{itemize}
  \begin{figure}
    \centering
    \includegraphics[width=0.7\textwidth]{../../../images/imgfinals/dialogue_ouvrir_projet.png}
    \caption{Dialogue d'ouverture et de création de projet}
  \end{figure}
\end{frame}

\begin{frame}{Parcours de l'Analyste (2/4)}
  \frametitle{2. Configuration du Modèle}
  \textbf{Dialogue de Configuration Central :}
  \begin{itemize}
    \item Définition des \textbf{sources de données} (SQL, API REST).
    \item Création des \textbf{entités du flux} (étapes, fournisseurs, clients).
    \item Définition de la \textbf{séquence logique} du flux.
    \item Attachement des \textbf{indicateurs} avec mode de collecte.
  \end{itemize}
  \vfill
  \textbf{Approche « Model-First » :} On modélise, on ne dessine pas.
\end{frame}

\begin{frame}{Parcours de l'Analyste (2/4 suite)}
  \frametitle{Captures : Configuration du Modèle}
  \begin{columns}[c]
    \begin{column}{0.48\textwidth}
      \centering
      \includegraphics[width=\textwidth]{../../../images/imgfinals/dialconfig_vu_infos_general.png}
      \tiny{Infos générales}
    \end{column}
    \begin{column}{0.48\textwidth}
      \centering
      \includegraphics[width=\textwidth]{../../../images/imgfinals/dialconfig_vu_etapes_de_production.png}
      \tiny{Étapes de production}
    \end{column}
  \end{columns}
  \vfill
  \begin{center}
    \includegraphics[width=0.45\textwidth]{../../../images/imgfinals/dialconfig_vu_indicteurs.png}
    \tiny{Gestion des indicateurs}
  \end{center}
\end{frame}

\begin{frame}{Parcours de l'Analyste (3/4)}
  \frametitle{3. Visualisation et Analyse de l'État Actuel}
  \textbf{Après configuration :} Diagramme généré automatiquement.
  \begin{itemize}
    \item Navigation dans le diagramme.
    \item Analyse des \textbf{indicateurs globaux calculés} (Lead Time, \% VA).
    \item \textbf{Règles de détection intelligentes} : goulots, gaspillages surlignés.
    \item Interaction avec l'\textbf{assistant IA}.
  \end{itemize}
\end{frame}

\begin{frame}{Parcours de l'Analyste (3/4 suite)}
  \frametitle{Captures : Analyse de l'État Actuel}
  \begin{figure}
    \centering
    \includegraphics[width=0.9\textwidth]{../../../images/imgfinals/vu_diagramme_etat_actuel_avec_panneaux_explorateur_et_propriete.png}
    \caption{Vue globale avec panneaux latéraux}
  \end{figure}
\end{frame}

\begin{frame}{Parcours de l'Analyste (3/4 suite)}
  \frametitle{Détection des Problèmes}
  \begin{figure}
    \centering
    \includegraphics[width=0.8\textwidth]{../../../images/imgfinals/vu_avec_panneau_analyse_montrant_10_probleme.png}
    \caption{Panneau d'analyse : 10 problèmes détectés}
  \end{figure}
\end{frame}

\begin{frame}{Parcours de l'Analyste (3/4 suite)}
  \frametitle{Assistant Conversationnel}
  \begin{figure}
    \centering
    \includegraphics[width=0.35\textwidth]{../../../images/imgfinals/vu_avec_panneau_assistant.png}
    \caption{L'assistant IA intégré}
  \end{figure}
\end{frame}

\begin{frame}{Parcours de l'Analyste (4/4)}
  \frametitle{4. Conception de l'État Futur}
  \begin{itemize}
    \item \textbf{Duplication} du diagramme état actuel.
    \item Modélisation des \textbf{améliorations} envisagées.
    \item \textbf{Mise à jour instantanée} des indicateurs globaux.
    \item \textbf{Mesure de l'efficacité} avant déploiement terrain.
  \end{itemize}
  \begin{figure}
    \centering
    \includegraphics[width=0.6\textwidth]{../../../images/imgfinals/dial_cration_diagramme_etat_futur.png}
    \caption{Création d'un diagramme État Futur}
  \end{figure}
\end{frame}

\begin{frame}{Parcours de l'Analyste (5/5)}
  \frametitle{5. Enrichissement et Plan d'Action}
  \begin{itemize}
    \item Ajout de \textbf{points d'amélioration} (Kaizen) sur le diagramme.
    \item Rédaction de \textbf{notes} contextuelles.
    \item Définition d'un \textbf{plan d'action} structuré.
    \item Export au format \texttt{.vsmx}.
  \end{itemize}
  \vfill
  Tout est stocké dans la base de données.
\end{frame}

\begin{frame}{Parcours de l'Analyste (5/5 suite)}
  \frametitle{Captures : Plan d'Action et Notes}
  \begin{columns}[c]
    \begin{column}{0.48\textwidth}
      \centering
      \includegraphics[width=\textwidth]{../../../images/imgfinals/vu_avec_panneau_plan_daction_montrant_une_action_a_realiser.png}
      \tiny{Gestion du plan d'action}
    \end{column}
    \begin{column}{0.48\textwidth}
      \centering
      \includegraphics[width=\textwidth]{../../../images/imgfinals/vu_note_prise.png}
      \tiny{Prise de notes contextuelle}
    \end{column}
  \end{columns}
\end{frame}

\begin{frame}{Configuration Indicateur Dynamique}
  \frametitle{Focus : Mode Dynamique en 2 Étapes}
  \textbf{Étape 1 : Définition des sources de données}
  \begin{itemize}
    \item Bibliothèque de connexions (SQL, API REST).
    \item Informations de connexion centralisées et réutilisables.
  \end{itemize}
  \vfill
  \textbf{Étape 2 : Liaison de l'indicateur}
  \begin{itemize}
    \item Sélection de la source de données.
    \item Configuration contextuelle :
          \begin{itemize}
            \item Si SQL : Requête SQL personnalisée.
            \item Si API : Endpoint spécifique.
          \end{itemize}
  \end{itemize}
  \vfill
  \textit{Configuration claire, sécurisée et flexible.}
\end{frame}

\begin{frame}{Parcours de l'Opérateur}
  \frametitle{Collecte de Données Terrain}
  \textbf{Interface Web simplifiée :}
  \begin{enumerate}
    \item Accès via navigateur depuis l'atelier.
    \item Sélection du poste / processus.
    \item Saisie des valeurs d'indicateurs (mode manuel).
    \item Validation $\rightarrow$ Synchronisation instantanée avec la base de données.
  \end{enumerate}
  \vfill
  \textbf{Objectif :} Minimiser l'impact sur les tâches de production.
\end{frame}

\begin{frame}{Parcours du Manager}
  \frametitle{Supervision et Prise de Décision}
  \textbf{Dashboard de visualisation (Web) :}
  \begin{enumerate}
    \item Accès via navigateur (bureau ou salle de réunion).
    \item Consultation du diagramme VSM en \textbf{lecture seule}.
    \item Indicateurs rafraîchis en \textbf{temps réel}.
    \item Alertes visuelles (goulots surlignés).
  \end{enumerate}
  \vfill
  \textbf{Résultat :} Décisions éclairées pour piloter l'activité.
\end{frame}

% ==============================================================================
% SECTION 7 : POSITIONNEMENT CONCURRENTIEL
% ==============================================================================
\section{Positionnement Concurrentiel}

\begin{frame}{Panorama des Solutions Existantes}
  \frametitle{Le Marché des Outils VSM}
  \textbf{3 grandes familles d'outils :}
  \begin{enumerate}
    \item \textbf{Outils de dessin génériques}
          \begin{itemize}
            \item Ex: Visio, Lucidchart, Draw.io
            \item Forces : Flexibles, accessibles
            \item Limites : Aucune intelligence métier, pas de calculs auto
          \end{itemize}
    \item \textbf{Outils VSM spécialisés}
          \begin{itemize}
            \item Ex: visTABLE®, Simcad Pro
            \item Forces : Calculs intégrés, simulation
            \item Limites : Complexes, coûteux, parfois surdimensionnés
          \end{itemize}
    \item \textbf{Plateformes VSM Management}
          \begin{itemize}
            \item Ex: Planview, Axify
            \item Forces : Intégration d'entreprise, IA
            \item Limites : Très coûteuses, abstraites, focus logiciel
          \end{itemize}
  \end{enumerate}
\end{frame}

\begin{frame}{Comparaison avec les Acteurs Clés}
  \frametitle{Positionnement de VSM-Tools}
  \begin{itemize}
    \item \textbf{vs. visTABLE® :} Excellence en Factory Layout, moins centré sur les indicateurs Lean globaux.
    \item \textbf{vs. Simcad Pro :} Simulation d'événements discrets très puissante, mais complexe et coûteuse.
    \item \textbf{vs. Lucidchart :} Collaboratif et basé web, mais reste un outil de dessin sans connexion de données.
  \end{itemize}
\end{frame}

\begin{frame}{Nos Atouts Différenciants}
  \frametitle{Ce qui Rend VSM-Tools Unique}
  \begin{itemize}
    \item \textbf{Approche « Model-First » unique :}
          \begin{itemize}
            \item Guidage, cohérence, fiabilité.
          \end{itemize}
    \item \textbf{Flexibilité inégalée de collecte de données :}
          \begin{itemize}
            \item 3 modes (statique, manuel, dynamique).
            \item S'adapte à la maturité digitale.
          \end{itemize}
    \item \textbf{Intelligence accessible :}
          \begin{itemize}
            \item Suggestions, détection, assistant IA.
            \item Sans submerger l'utilisateur.
          \end{itemize}
    \item \textbf{Architecture ouverte et moderne :}
          \begin{itemize}
            \item Technologies répandues (Electron, Node.js, React).
            \item Multi-plateformes, évolutive.
          \end{itemize}
  \end{itemize}
  \vfill
  \centering
  \textit{La solution la plus pragmatique et adaptable.}
\end{frame}

% ==============================================================================
% SECTION 8 : ÉCOSYSTÈME INDUSTRIEL
% ==============================================================================
\section{Écosystème Industriel}

\begin{frame}{Pyramide d'Automatisation}
  \frametitle{Le Cadre de Référence}
  \begin{figure}
    \centering
    \includegraphics[width=0.65\textwidth]{../../../images/automation_pyramide.png}
    \caption{La pyramide d'automatisation}
  \end{figure}

  \textbf{Niveaux :} ERP (Stratégique) → MES (Pilotage) → SCADA (Supervision) → PLC (Contrôle) → Terrain (Capteurs)
\end{frame}

\begin{frame}{Place de VSM-Tools}
  \frametitle{Un Outil d'Analyse Transverse}
  \textbf{VSM-Tools n'est pas dans une couche unique :}
  \begin{itemize}
    \item C'est une \textbf{couche d'analyse et d'aide à la décision}.
    \item Interagit avec plusieurs niveaux.
  \end{itemize}
  \vfill
  \textbf{Son rôle :}
  \begin{itemize}
    \item \textbf{Consomme} : Données de l'ERP (planning) et du MES (production réelle).
    \item \textbf{Produit} : Analyses, objectifs d'amélioration, règles de détection.
    \item \textbf{Influence} : Stratégie (ERP) et actions d'optimisation (MES).
  \end{itemize}
  \vfill
  \textit{Focus : Compréhension et amélioration des processus transversaux.}
\end{frame}

\begin{frame}{Synergies et Interactions}
  \frametitle{VSM-Tools dans l'Écosystème}
  \begin{itemize}
    \item \textbf{Alimentation en données fiables :}
          \begin{itemize}
            \item Connecteurs vers MES/ERP.
            \item VSM « vivante » et précise.
          \end{itemize}
    \item \textbf{Orientation des actions :}
          \begin{itemize}
            \item Objectifs issus de l'analyse VSM.
            \item Paramétrage affiné des systèmes MES/ERP.
          \end{itemize}
    \item \textbf{Complémentarité fondamentale :}
          \begin{itemize}
            \item ERP : Gestion globale des ressources.
            \item MES : Exécution et contrôle quotidien.
            \item VSM-Tools : Comprendre, analyser, concevoir l'amélioration.
          \end{itemize}
  \end{itemize}
  \vfill
  \textit{Boucle vertueuse : Analyse → Action → Données → Analyse.}
\end{frame}

% ==============================================================================
% SECTION 9 : DÉFIS ET LIMITES
% ==============================================================================
\section{Défis et Limites}

\begin{frame}{Défis et Limites Actuelles}
  \frametitle{Points de Vigilance}
  \begin{itemize}
    \item \textbf{Dépendance à la qualité des données :}
          \begin{itemize}
            \item Pertinence = qualité, complétude, fiabilité des données d'entrée.
          \end{itemize}
    \item \textbf{Outil au service d'une démarche :}
          \begin{itemize}
            \item Ne remplace pas la compréhension Lean/VSM.
            \item Besoin d'équipes formées à la méthodologie.
          \end{itemize}
    \item \textbf{Sécurité et droits d'accès :}
          \begin{itemize}
            \item Authentification et ACL non implémentés (version actuelle).
            \item Prérequis pour déploiement à grande échelle.
          \end{itemize}
    \item \textbf{Périmètre fonctionnel actuel :}
          \begin{itemize}
            \item Contrainte : 1 fournisseur + 1 client par diagramme.
            \item Pas de simulation/prédiction avancée.
            \item Connectivité générique (SQL/API), pas de connecteurs métiers « clés en main ».
            \item Collaboration asynchrone uniquement.
          \end{itemize}
  \end{itemize}
\end{frame}

% ==============================================================================
% SECTION 10 : PERSPECTIVES ET ÉVOLUTIONS
% ==============================================================================
\section{Perspectives et Évolutions}

\begin{frame}{Enrichissement de l'Analyse}
  \frametitle{Vers Plus d'Intelligence}
  \begin{itemize}
    \item \textbf{Simulation « What-if » :}
          \begin{itemize}
            \item Modifier paramètres et simuler l'impact.
          \end{itemize}
    \item \textbf{Analyse prédictive :}
          \begin{itemize}
            \item Anticiper goulots futurs (Machine Learning).
          \end{itemize}
    \item \textbf{Tableau de bord comparatif :}
          \begin{itemize}
            \item Actuel vs. Futur avec calculs de gains automatisés.
          \end{itemize}
  \end{itemize}
\end{frame}

\begin{frame}{Connectivité et Intégrations}
  \frametitle{Faciliter la Connexion aux Données}
  \begin{itemize}
    \item \textbf{Bibliothèque de connecteurs métiers :}
          \begin{itemize}
            \item Connecteurs pré-configurés (SAP, Wonderware...).
            \item Réduction de l'effort de configuration manuelle.
          \end{itemize}
    \item \textbf{Intégrations natives :}
          \begin{itemize}
            \item Synchronisation automatique avec ERP/MES.
          \end{itemize}
  \end{itemize}
\end{frame}

\begin{frame}{Extension de la Modélisation}
  \frametitle{Flux Complexes et Collaboration}
  \begin{itemize}
    \item \textbf{Gestion des flux complexes :}
          \begin{itemize}
            \item Plusieurs fournisseurs et/ou clients.
            \item Flux avec branches (divergence/convergence).
          \end{itemize}
    \item \textbf{Collaboration temps réel :}
          \begin{itemize}
            \item Plusieurs utilisateurs sur une même cartographie simultanément.
          \end{itemize}
    \item \textbf{Gestion avancée des droits :}
          \begin{itemize}
            \item ACL (rôles : lecteur, éditeur, administrateur).
            \item Historique et versions.
          \end{itemize}
  \end{itemize}
\end{frame}

\begin{frame}{Évolution du Déploiement}
  \frametitle{Vers le Cloud / SaaS}
  \begin{itemize}
    \item \textbf{Offre Cloud / SaaS :}
          \begin{itemize}
            \item Version entièrement hébergée.
            \item Accès via navigateur, sans installation.
            \item Mises à jour automatiques.
            \item Modèle d'abonnement flexible.
          \end{itemize}
  \end{itemize}
  \vfill
  \textbf{Objectif :} Rendre VSM-Tools encore plus prédictif, intégré et collaboratif.
\end{frame}

% ==============================================================================
% SECTION 11 : MODÈLE ÉCONOMIQUE
% ==============================================================================
\section{Modèle Économique}

\begin{frame}{Segments de Clientèle}
  \frametitle{Trois Marchés Cibles}
  \begin{enumerate}
    \item \textbf{Individuels et Milieu Académique}
          \begin{itemize}
            \item Étudiants, enseignants, chercheurs, consultants indépendants.
          \end{itemize}
    \item \textbf{PME et Équipes d'Amélioration Continue}
          \begin{itemize}
            \item Petites et moyennes entreprises.
            \item Départements spécifiques dans les grands groupes.
          \end{itemize}
    \item \textbf{Grandes Entreprises}
          \begin{itemize}
            \item Besoins avancés : intégration, sécurité, collaboration, support.
          \end{itemize}
  \end{enumerate}
\end{frame}

\begin{frame}{Stratégie de Tarification}
  \frametitle{Modèle Freemium à 3 Niveaux}
  \begin{enumerate}
    \item \textbf{VSM-Tools Community (Gratuit)}
          \begin{itemize}
            \item 3 projets max, mode statique uniquement.
            \item Cible : Étudiants, enseignants.
          \end{itemize}
    \item \textbf{VSM-Tools Professional (Abonnement)}
          \begin{itemize}
            \item Projets illimités, tous les modes de collecte.
            \item Suggestions IA, dashboard web, support email.
            \item \textbf{~49€/utilisateur/mois}
          \end{itemize}
    \item \textbf{VSM-Tools Enterprise (Devis)}
          \begin{itemize}
            \item SSO, ACL, support prioritaire (SLA).
            \item On-Premise ou Cloud privé.
            \item Connecteurs métiers personnalisés.
          \end{itemize}
  \end{enumerate}
\end{frame}

\begin{frame}{Business Model Canvas}
  \frametitle{Synthèse du Modèle Économique}
  \begin{figure}
    \centering
    \includegraphics[width=0.85\textwidth]{../../../images/business_model_canvas.png}
    \caption{Business Model Canvas pour VSM-Tools}
  \end{figure}
\end{frame}

% ==============================================================================
% SECTION 12 : DÉMONSTRATION
% ==============================================================================
\section{Démonstration}

\begin{frame}{Démonstration Live}
  \frametitle{Passage à la Démonstration}
  \centering
  \vspace{2cm}
  {\LARGE \textbf{Démonstration de l'application}}
  \vspace{1cm}

  \begin{itemize}
    \item Création d'un projet VSM
    \item Configuration d'un flux de production
    \item Analyse des indicateurs et détection des goulots
    \item Conception d'un état futur optimisé
  \end{itemize}
\end{frame}

% ==============================================================================
% CONCLUSION
% ==============================================================================
\section{Conclusion}

\begin{frame}{Synthèse du Projet}
  \frametitle{Ce que nous avons réalisé}
  \begin{itemize}
    \item \textbf{Problème identifié :} Limites des méthodes VSM traditionnelles.
    \item \textbf{Solution développée :} VSM-Tools, une application complète et moderne.
    \item \textbf{Apports clés :}
          \begin{itemize}
            \item Simplification et fiabilisation du processus VSM.
            \item Dynamisation des données (3 modes de collecte).
            \item Analyse augmentée par l'intelligence artificielle.
            \item Architecture moderne et évolutive.
          \end{itemize}
    \item \textbf{Résultat :} Un outil pragmatique et adaptable pour l'amélioration continue.
  \end{itemize}
\end{frame}

\begin{frame}{Perspectives}
  \frametitle{L'Avenir de VSM-Tools}
  \begin{itemize}
    \item \textbf{Court terme :}
          \begin{itemize}
            \item Implémentation de l'authentification et ACL.
            \item Enrichissement de la bibliothèque de connecteurs.
          \end{itemize}
    \item \textbf{Moyen terme :}
          \begin{itemize}
            \item Simulation « What-if » et analyse prédictive.
            \item Support des flux complexes (multi-fournisseurs/clients).
          \end{itemize}
    \item \textbf{Long terme :}
          \begin{itemize}
            \item Offre Cloud / SaaS.
            \item Collaboration temps réel.
          \end{itemize}
  \end{itemize}
  \vfill
  \textbf{Vision :} Devenir la plateforme de référence pour l'analyse VSM.
\end{frame}

\begin{frame}{Impact et Valeur}
  \frametitle{L'Apport de VSM-Tools à l'Industrie}
  \begin{itemize}
    \item \textbf{Pour les entreprises :}
          \begin{itemize}
            \item Gains de temps et de productivité.
            \item Décisions basées sur des données fiables.
            \item Amélioration continue facilitée.
          \end{itemize}
    \item \textbf{Pour le secteur :}
          \begin{itemize}
            \item Démocratisation de l'analyse VSM.
            \item Digitalisation des démarches Lean.
            \item Pont entre méthodes traditionnelles et Industrie 4.0.
          \end{itemize}
  \end{itemize}
  \vfill
  \centering
  \textit{VSM-Tools : L'intelligence au service de l'excellence opérationnelle.}
\end{frame}

% --- Diapositive de fin ---
\begin{frame}
  \centering
  {\Huge Merci de votre attention}
  \vspace{1cm}

  \textbf{Questions ?}
\end{frame}

\end{document}
