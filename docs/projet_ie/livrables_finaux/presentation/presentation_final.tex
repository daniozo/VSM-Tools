\documentclass{beamer}

% --- Thème ---
\usetheme{metropolis} % Thème moderne et épuré

% Supprimer les symboles de navigation
\setbeamertemplate{navigation symbols}{}

% --- Packages utiles ---
\usepackage[utf8]{inputenc}
\usepackage[T1]{fontenc}
\usepackage{lmodern}
\usepackage[french, provide=*]{babel} % Pour la typographie française
\usepackage{graphicx} % Pour inclure des images
\usepackage{amsmath} % Pour les maths
\usepackage{booktabs} % Pour de jolis tableaux
\usepackage{hyperref} % Pour les liens
\usepackage{xurl}     % Pour mieux couper les URLs
\usepackage{tikz}     % Pour positionner le logo

% --- Informations pour la page de titre ---
\title[VSM-Tools]{Développement d'une application de visualisation des flux de production pour identifier les goulots d'étranglement et optimiser les performances
industrielles}
\subtitle{Présentation Finale}
\author{MPOKE Jonathan, PAMBO PAGA Chris Jeredh,\\
        ZIDA Eben-Ezer, ZOUGOU TOVIGNON Comlan Daniel}
\institute{EIGSI}
\date{2025 - 2026}

% --- Logo sur chaque slide (coin supérieur droit) ---
\addtobeamertemplate{frametitle}{}{%
  \begin{tikzpicture}[remember picture,overlay]
    \node[anchor=north east,yshift=2pt] at (current page.north east) {%
      \includegraphics[height=0.8cm]{../../../images/logo_eigsi.png}%
    };
  \end{tikzpicture}%
}

% Pagination simple : numéro de slide seulement
\setbeamertemplate{footline}{%
  \leavevmode%
  \hbox{%
    \begin{beamercolorbox}[wd=\paperwidth,ht=2.5ex,dp=1.125ex,right]{author in head/foot}%
      \usebeamerfont{page number in head/foot}\insertframenumber\hspace{0.5em}
    \end{beamercolorbox}%
  }%
}

\begin{document}

% --- Page de Titre ---
\begin{frame}
  \centering
  \vspace{0.3cm}
  \includegraphics[width=0.22\textwidth]{../../../images/logo_eigsi.png}
  \vspace{0.3cm}
  \titlepage
\end{frame}

% --- Table des Matières ---
\begin{frame}{Plan de la présentation}
  \tableofcontents
\end{frame}

% ==============================================================================
% SECTION 1 : INTRODUCTION (2 slides)
% ==============================================================================
\section{Introduction}

\begin{frame}{Contexte et Objectifs}
  \frametitle{Introduction : Le Défi de l'Industrie Moderne}
  \begin{columns}[T]
    \begin{column}{0.5\textwidth}
      \textbf{Contexte :}
      \begin{itemize}
        \item Quête perpétuelle d'\textbf{efficacité}
        \item Compétition mondiale
        \item Exigences clients croissantes
      \end{itemize}
    \end{column}
    \begin{column}{0.5\textwidth}
      \textbf{Notre réponse :}
      \begin{itemize}
        \item Solution logicielle dédiée
        \item Basée sur le \textbf{VSM}
        \item Analyse et optimisation des flux
      \end{itemize}
    \end{column}
  \end{columns}
  \vfill
  \textbf{Ce que nous allons présenter :}
  \begin{enumerate}
    \item Enjeux industriels et méthodologie VSM
    \item VSM-Tools : architecture et fonctionnalités
    \item Positionnement, modèle économique et perspectives
  \end{enumerate}
\end{frame}

% ==============================================================================
% SECTION 2 : LES ENJEUX DE L'OPTIMISATION INDUSTRIELLE (2 slides)
% ==============================================================================
\section{Les enjeux de l'optimisation industrielle}

\begin{frame}{Contexte et Piliers de la Performance}
  \frametitle{Les Défis et le Triptyque QCD}
  \begin{columns}[T]
    \begin{column}{0.5\textwidth}
      \textbf{Pressions continues :}
      \begin{itemize}
        \item Concurrence mondiale
        \item Exigence clients
        \item Volatilité des marchés
        \item Pression sur les ressources
      \end{itemize}
    \end{column}
    \begin{column}{0.5\textwidth}
      \textbf{Triptyque QCD :}
      \begin{itemize}
        \item \textbf{Qualité} : Zéro défaut
        \item \textbf{Coûts} : Éliminer les gaspillages
        \item \textbf{Délais} : Réduire le Lead Time
      \end{itemize}
    \end{column}
  \end{columns}
  \vfill
  \centering
  \textit{L'optimisation n'est plus un confort, c'est une condition de survie.}
\end{frame}

\begin{frame}{Le Défi de la Visibilité}
  \frametitle{De la Donnée à la Décision}
  \textbf{Principal obstacle :} Le manque de \textbf{visibilité}.
  \begin{itemize}
    \item Processus = systèmes complexes avec dépendances cachées
    \item Gaspillages et goulots \textbf{noyés dans la complexité}
    \item Efforts d'amélioration locaux, parfois contre-productifs
  \end{itemize}
  \vfill
  \centering
  {\Large \textbf{Pour optimiser, il faut d'abord VOIR.}}
  \vfill
  \textit{C'est la raison d'être du Value Stream Mapping.}
\end{frame}

% ==============================================================================
% SECTION 3 : LE VALUE STREAM MAPPING (2 slides)
% ==============================================================================
\section{Le Value Stream Mapping}

\begin{frame}{Le VSM : Définition et Importance}
  \frametitle{Cartographie des Flux de Valeur}
  \begin{columns}[T]
    \begin{column}{0.55\textwidth}
      \textbf{Qu'est-ce que le VSM ?}
      \begin{itemize}
        \item Méthode visuelle du \textbf{Lean Management}
        \item Cartographie flux matière + information
        \item Du Fournisseur au Client
        \item \textbf{Toutes} les étapes (VA ou non)
      \end{itemize}
      \vspace{0.3cm}
      \textbf{Pourquoi c'est essentiel :}
      \begin{itemize}
        \item Identifier gaspillages et goulots
        \item Réduire délais et coûts
        \item Base de l'amélioration continue
      \end{itemize}
    \end{column}
    \begin{column}{0.45\textwidth}
      \centering
      \includegraphics[width=0.9\textwidth]{../../../images/vsm_etat_actuel}
      \tiny{Exemple de VSM État Actuel}
    \end{column}
  \end{columns}
\end{frame}

\begin{frame}{Limites des Méthodes Traditionnelles}
  \frametitle{Le Besoin d'un Outil Dédié}
  \begin{columns}[T]
    \begin{column}{0.5\textwidth}
      \textbf{Outils actuels :}
      \begin{itemize}
        \item \textbf{Papier/Post-it :} Difficile à maintenir
        \item \textbf{Excel :} Pas de visualisation
        \item \textbf{Visio/Draw.io :} Pas d'intelligence métier
      \end{itemize}
      \vspace{0.3cm}
      \textbf{Problèmes communs :}
      \begin{itemize}
        \item Aucun calcul automatique
        \item Pas de connexion aux données
        \item Cartes \textbf{statiques}
      \end{itemize}
    \end{column}
    \begin{column}{0.5\textwidth}
      \centering
      \includegraphics[width=0.9\textwidth]{../../../images/vsm_etat_futur}
      \tiny{Exemple de VSM État Futur}
    \end{column}
  \end{columns}
  \vfill
  \centering
  \textit{Besoin d'un outil simple, puissant et connecté aux données réelles.}
\end{frame}

% ==============================================================================
% SECTION 4 : VSM-TOOLS : NOTRE SOLUTION (2 slides)
% ==============================================================================
\section{VSM-Tools : Notre Solution}

\begin{frame}{Mission et Apports Clés}
  \frametitle{VSM-Tools : Notre Réponse aux Défis}
  \textbf{Mission :} Rendre l'analyse VSM \textbf{accessible, rapide et actionnable}.
  \vfill
  \textbf{Les 4 apports clés :}
  \begin{enumerate}
    \item \textbf{Simplification :} Modélisation guidée, calculs automatiques
    \item \textbf{Dynamisation :} 3 modes de collecte (Statique, Manuel, Dynamique SQL/API)
    \item \textbf{Intelligence :} Détection goulots, suggestions IA, assistant conversationnel
    \item \textbf{Architecture moderne :} Studio + Moteur (Server) + Interfaces Web (Interfaces opérateur et Dashboard de visualisation)
  \end{enumerate}
\end{frame}

\begin{frame}{Cycle VSM et Gestion de Projet}
  \frametitle{État Actuel → État Futur}
  \begin{columns}[T]
    \begin{column}{0.5\textwidth}
      \textbf{Gestion du cycle VSM :}
      \begin{itemize}
        \item Duplication état actuel → futur
        \item Modélisation des améliorations
        \item Mise à jour instantanée des KPIs
        \item Quantification de l'impact
      \end{itemize}
    \end{column}
    \begin{column}{0.5\textwidth}
      \textbf{Gestion de projet intégrée :}
      \begin{itemize}
        \item Plan d'action structuré
        \item Notes contextuelles
        \item Import/Export \texttt{.vsmx}
        \item Base de données centralisée
      \end{itemize}
    \end{column}
  \end{columns}
\end{frame}

% ==============================================================================
% SECTION 5 : ARCHITECTURE TECHNIQUE (2 slides)
% ==============================================================================
\section{Architecture Technique}

\begin{frame}{Vue d'Ensemble}
  \frametitle{Architecture Modulaire et Découplée}
  \begin{figure}
    \centering
    \includegraphics[width=0.7\textwidth]{../../../images/architecture_vsm_tools}
  \end{figure}
  \textbf{3 composants principaux:} VSM Studio • VSM Engine • Base de données
\end{frame}

\begin{frame}{Composants et Interfaces}
  \frametitle{Technologies et Responsabilités}
  \begin{columns}[T]
    \begin{column}{0.33\textwidth}
      \textbf{VSM Studio}
      \begin{itemize}
        \item Electron (multi-OS)
        \item Interface principale
        \item Modélisation
        \item Assistant IA
      \end{itemize}
    \end{column}
    \begin{column}{0.33\textwidth}
      \textbf{VSM Engine}
      \begin{itemize}
        \item Node.js
        \item Logique métier
        \item Calculs auto
        \item Connecteurs SQL/API
        \item API REST
      \end{itemize}
    \end{column}
    \begin{column}{0.33\textwidth}
      \textbf{Interfaces Web}
      \begin{itemize}
        \item React
        \item Saisie opérateur
        \item Dashboard manager
        \item Temps réel
      \end{itemize}
    \end{column}
  \end{columns}
  \vfill
  \textbf{PostgreSQL :} Centralisation des données, scalabilité, accès unifié via Engine.
\end{frame}

% ==============================================================================
% SECTION 6 : PARCOURS UTILISATEUR (1 slide)
% ==============================================================================
\section{Parcours Utilisateur}

\begin{frame}{Parcours de l'Analyste}
  \frametitle{De la Configuration à l'Action}
  \textbf{Étapes du parcours :}
  \begin{enumerate}
    \item Initialisation projet
    \item Configuration modèle
    \item Visualisation état actuel
    \item Analyse et détection goulots
    \item Conception état futur
    \item Plan d'action et export
  \end{enumerate}
  \vfill
  \textbf{Trois acteurs :} Analyste (Studio) • Opérateur (Web saisie) • Manager (Dashboard)
\end{frame}

% ==============================================================================
% SECTION 7 : POSITIONNEMENT CONCURRENTIEL (2 slides)
% ==============================================================================
\section{Positionnement Concurrentiel}

\begin{frame}{Panorama et Comparaison}
  \frametitle{Le Marché des Outils VSM}
  \begin{columns}[T]
    \begin{column}{0.5\textwidth}
      \textbf{3 familles d'outils :}
      \begin{itemize}
        \item \textbf{Dessin générique} (Visio, Lucidchart)\\
              Flexibles mais sans intelligence métier
        \item \textbf{VSM spécialisés} (visTABLE, Simcad)\\
              Puissants mais complexes et coûteux
        \item \textbf{Plateformes VSM} (Planview)\\
              Complets mais très chers, focus logiciel
      \end{itemize}
    \end{column}
    \begin{column}{0.5\textwidth}
      \textbf{Nos atouts différenciants :}
      \begin{itemize}
        \item Approche guidée et simplifiée, étape par étape
        \item 3 modes de collecte de données
        \item Intelligence accessible (IA)
        \item Architecture ouverte moderne
        \item Multi-plateformes
      \end{itemize}
    \end{column}
  \end{columns}
  \vfill
\end{frame}

% ==============================================================================
% SECTION 8 : ÉCOSYSTÈME INDUSTRIEL (2 slides)
% ==============================================================================
\section{Écosystème Industriel}

\begin{frame}{Place dans l'Écosystème}
  \frametitle{VSM-Tools et la Pyramide d'Automatisation}
  \begin{columns}[T]
    \begin{column}{0.45\textwidth}
      \includegraphics[width=\textwidth]{../../../images/automation_pyramide.png}
    \end{column}
    \begin{column}{0.55\textwidth}
      \textbf{VSM-Tools = Couche d'analyse transverse}
      \vspace{0.1cm}
      \begin{itemize}
        \item \textbf{Consomme :} Données ERP + MES
        \item \textbf{Produit :} Analyses, objectifs, règles
        \item \textbf{Influence :} Stratégie et optimisation
      \end{itemize}
      \textbf{Complémentarité :}
      \begin{itemize}
        \item ERP : Gestion globale
        \item MES : Exécution quotidienne
        \item VSM-Tools : Comprendre et améliorer
      \end{itemize}
    \end{column}
  \end{columns}
\end{frame}

% ==============================================================================
% SECTION 9 : DÉFIS ET LIMITES (1 slide)
% ==============================================================================
\section{Défis et Limites}

\begin{frame}{Défis et Limites Actuelles}
  \frametitle{Points de Vigilance}
  \begin{columns}[T]
    \begin{column}{0.5\textwidth}
      \textbf{Dépendances :}
      \begin{itemize}
        \item Qualité des données d'entrée
        \item Compréhension Lean/VSM requise
        \item Équipes formées
      \end{itemize}
      \vspace{0.3cm}
      \textbf{Non implémenté :}
      \begin{itemize}
        \item Authentification et ACL
        \item Simulation/Prédiction avancée
      \end{itemize}
    \end{column}
    \begin{column}{0.5\textwidth}
      \textbf{Contraintes actuelles :}
      \begin{itemize}
        \item 1 fournisseur + 1 client par VSM
        \item Connectivité générique (SQL/API)
        \item Collaboration asynchrone
      \end{itemize}
      \vspace{0.3cm}
      \textit{Ces limites définissent la feuille de route des évolutions futures.}
    \end{column}
  \end{columns}
\end{frame}

% ==============================================================================
% SECTION 10 : PERSPECTIVES ET ÉVOLUTIONS (2 slides)
% ==============================================================================
\section{Perspectives et Évolutions}

\begin{frame}{Évolutions Futures}
  \frametitle{La Feuille de Route}
  \begin{columns}[T]
    \begin{column}{0.5\textwidth}
      \textbf{Analyse enrichie :}
      \begin{itemize}
        \item Simulation « What-if »
        \item Analyse prédictive (ML)
        \item Comparatif Actuel vs Futur
      \end{itemize}
      \vspace{0.3cm}
      \textbf{Connectivité :}
      \begin{itemize}
        \item Connecteurs métiers (SAP...)
        \item Intégrations natives ERP/MES
      \end{itemize}
    \end{column}
    \begin{column}{0.5\textwidth}
      \textbf{Modélisation :}
      \begin{itemize}
        \item Multi-fournisseurs/clients
        \item Flux avec branches
        \item Collaboration temps réel
      \end{itemize}
      \vspace{0.3cm}
      \textbf{Déploiement :}
      \begin{itemize}
        \item Offre Cloud / SaaS
        \item Abonnement flexible
      \end{itemize}
    \end{column}
  \end{columns}
  \vfill
  \centering
  \textbf{Objectif :} Plus prédictif, intégré et collaboratif.
\end{frame}

% ==============================================================================
% SECTION 11 : MODÈLE ÉCONOMIQUE (2 slides)
% ==============================================================================
\section{Modèle Économique}

\begin{frame}{Stratégie de Tarification}
  \frametitle{Modèle Freemium à 3 Niveaux}
  \begin{columns}[T]
    \begin{column}{0.33\textwidth}
      \centering
      \textbf{Community}\\
      \textit{Gratuit}\\[0.3cm]
      \begin{itemize}
        \item 3 projets max
        \item Mode statique
        \item Étudiants, enseignants
      \end{itemize}
    \end{column}
    \begin{column}{0.33\textwidth}
      \centering
      \textbf{Professional}\\
      \textit{prix fixe/mois/user}\\[0.3cm]
      \begin{itemize}
        \item Projets illimités
        \item Tous les modes
        \item IA, dashboard
        \item PME, consultants
      \end{itemize}
    \end{column}
    \begin{column}{0.33\textwidth}
      \centering
      \textbf{Enterprise}\\
      \textit{Sur devis}\\[0.3cm]
      \begin{itemize}
        \item SSO, ACL
        \item Support SLA
        \item On-Premise/Cloud
        \item Grandes entreprises
      \end{itemize}
    \end{column}
  \end{columns}
\end{frame}
\begin{frame}{Business Model Canvas}
  \frametitle{Business Model Canvas}
  \begin{figure}
    \centering
    \includegraphics[width=0.9\textwidth]{../../../images/business_model_canvas.png}
  \end{figure}
\end{frame}

% ==============================================================================
% SECTION 12 : DÉMONSTRATION (1 slide)
% ==============================================================================
\section{Démonstration}

\begin{frame}{Démonstration Live}
  \frametitle{Passage à la Démonstration}
  \centering
  \vspace{1cm}
  {\LARGE \textbf{Démonstration de l'application}}
  \vspace{0.5cm}

  \begin{itemize}
    \item Chargement d'un projet VSM
    \item Commentaires sur le diagramme
    \item Analyse des indicateurs et détection des goulots
    \item Conception d'un état futur optimisé
  \end{itemize}
\end{frame}

% ==============================================================================
% CONCLUSION (2 slides)
% ==============================================================================
\section{Conclusion}

\begin{frame}{Synthèse et Perspectives}
  \frametitle{Ce que nous avons réalisé}
  \begin{columns}[T]
    \begin{column}{0.5\textwidth}
      \textbf{Réalisations :}
      \begin{itemize}
        \item Problème identifié : limites VSM traditionnel
        \item Solution complète et moderne
        \item 4 apports clés :
              \begin{itemize}
                \item Simplification
                \item Dynamisation
                \item Intelligence
                \item Architecture moderne
              \end{itemize}
      \end{itemize}
    \end{column}
    \begin{column}{0.5\textwidth}
      \textbf{Perspectives :}
      \begin{itemize}
        \item Court terme : Auth, ACL, connecteurs
        \item Moyen terme : Simulation, flux complexes
        \item Long terme : Cloud/SaaS, temps réel
      \end{itemize}
      \vspace{0.3cm}
      \textbf{Impact :}
      \begin{itemize}
        \item Démocratisation VSM
        \item Digitalisation Lean
        \item Pont vers Industrie 4.0
      \end{itemize}
    \end{column}
  \end{columns}
  \vfill
  \centering
  \textit{VSM-Tools : L'intelligence au service de l'excellence opérationnelle.}
\end{frame}

% --- Diapositive de fin ---
\begin{frame}
  \centering
  {\Huge Merci de votre attention}
  \vspace{1cm}

  \textbf{Questions ?}
\end{frame}

\end{document}
