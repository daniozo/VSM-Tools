\documentclass[11pt, a4paper]{article}

% --- PAQUETAGES ESSENTIELS ---
\usepackage[utf8]{inputenc}
\usepackage[T1]{fontenc}
\usepackage{lmodern}
\usepackage[french, provide=*]{babel} % Pour la typographie française (Table des matières, Figure...)

% --- GÉOMÉTRIE ET MISE EN PAGE ---
\usepackage{geometry}
\geometry{a4paper, left=2.5cm, right=2.5cm, top=2.5cm, bottom=2.5cm} % Définir les marges

% --- GRAPHIQUES ET FLOTTANTS ---
\usepackage{graphicx} % Requis pour inclure des images
\usepackage{float} % Pour un meilleur contrôle du placement des flottants (figures, tables)

% --- RÉFÉRENCES ET LIENS ---
\usepackage{hyperref} % Pour les liens cliquables et les références
\usepackage{natbib} % Pour la bibliographie
\usepackage{xurl} % Pour une meilleure coupure de ligne des URLs

% --- TYPOGRAPHIE ET LISTES ---
\usepackage{csquotes} % Pour les guillemets intelligents avec \enquote{}
\usepackage{textcomp} % Pour les symboles comme \textregistered
\usepackage{enumitem} % Pour personnaliser les listes

% --- MATHÉMATIQUES (si nécessaire) ---
\usepackage{amsmath}
\usepackage{amssymb}

\usepackage[table]{xcolor}
\usepackage{pgfgantt}

% --- CONFIGURATION DE HYPERREF ---
\hypersetup{
    colorlinks=true,
    linkcolor=blue,
    filecolor=magenta,
    urlcolor=blue,
    pdftitle={Rapport Final de Projet},
    pdfpagemode=FullScreen,
}

\begin{document}

% --- PAGE DE GARDE ---
\begin{titlepage}
    \centering % Centre tout le contenu de la page de garde

    % Logo (assurez-vous que le chemin est correct)
    \includegraphics[width=0.3\textwidth]{../../../images/logo_eigsi.png}\vspace{3cm}

    % Titre du rapport
    {\LARGE \bfseries Développement d'une application de visualisation des flux de production pour identifier les goulots d'étranglement et optimiser les performances industrielles}\vspace{1.5cm}

    {\huge \bfseries Rapport Final}\vspace{3cm}

    % Membres de l'équipe
    {\large \textbf{Membres de l'équipe :}}\vspace{0.5cm} \\
    MPOKE Jonathan \\
    PAMBO PAGA Chris Jeredh \\
    ZIDA Eben-Ezer \\
    ZOUGOU TOVIGNON Comlan Daniel \vspace{1.5cm}

    % Encadrants
    {\large \textbf{Encadrants :}}\vspace{0.5cm} \\
    ERROUSSO Hanae \\
    BAROUD Sohaib \vspace{3cm}

    % Année académique
    {\large Année académique : 2025 - 2026}

    \vfill % Pousse le contenu vers le haut et le bas si nécessaire

\end{titlepage}

% --- TABLE DES MATIÈRES ---
\tableofcontents
\newpage

% --- SECTION 1 : INTRODUCTION ---
\section{Introduction}

Ce document constitue le rapport final du projet VSM-Tools. Il a pour objectif de présenter de manière exhaustive la démarche qui a guidé la conception et le développement d'une solution logicielle dédiée à l'optimisation des performances industrielles. De l'analyse des enjeux stratégiques du secteur à la justification des choix architecturaux et technologiques finaux, ce rapport retrace l'ensemble du travail accompli.

L'industrie moderne, confrontée à une compétition mondiale et à des exigences clients croissantes, est engagée dans une quête perpétuelle d'efficacité. Dans ce contexte, la capacité à visualiser, comprendre et améliorer les processus internes n'est plus une option, mais une nécessité stratégique. C'est pour répondre à ce besoin fondamental que ce projet a été initié, en se concentrant sur une méthodologie reconnue pour sa puissance d'analyse : le Value Stream Mapping (VSM).

Ce rapport s'articulera autour de plusieurs axes majeurs. Il commencera par définir les enjeux globaux de l'optimisation industrielle pour poser le cadre de notre action. Il détaillera ensuite la méthodologie VSM, qui constitue le cœur fonctionnel de notre solution. Enfin, il présentera l'application VSM-Tools elle-même : sa mission, son architecture technique finale basée sur les technologies Electron et Node.js, sa proposition de valeur, et le parcours utilisateur que nous avons conçu.


% --- SECTION 2 : LES ENJEUX DE L'OPTIMISATION INDUSTRIELLE ---
\section{Les enjeux de l'optimisation industrielle}
Le développement de VSM-Tools s'inscrit dans une réponse directe aux défis pressants auxquels l'industrie moderne est confrontée. Avant de détailler la méthodologie VSM, qui est le cœur de notre solution, il est essentiel de poser le cadre général de son objectif final : l'optimisation industrielle. Loin d'être un simple objectif de confort, l'optimisation est aujourd'hui une condition de survie et de compétitivité.

\subsection{Un contexte de pression continue}
Les entreprises industrielles, quelle que soit leur taille, évoluent dans un environnement marqué par des pressions multiples et croissantes :
\begin{itemize}[leftmargin=*]
    \item \textbf{La concurrence mondiale :} La mondialisation impose une compétition accrue, où la capacité à produire mieux, plus vite et moins cher est un différenciant majeur.
    \item \textbf{L'exigence des clients :} Les clients attendent des produits de haute qualité, personnalisés, et livrés dans des délais toujours plus courts. La moindre défaillance peut entraîner une perte de confiance et de parts de marché.
    \item \textbf{La volatilité des marchés :} Les chaînes d'approvisionnement sont devenues complexes et fragiles. La capacité à s'adapter rapidement aux ruptures (géopolitiques, sanitaires, technologiques) est devenue une question de résilience.
    \item \textbf{La pression sur les ressources :} L'augmentation du coût des matières premières, de l'énergie, et les nouvelles réglementations environnementales obligent les entreprises à produire plus avec moins, en minimisant leur empreinte écologique.
\end{itemize}

\subsection{Les piliers de la performance opérationnelle}
Face à ce contexte, l'optimisation industrielle vise à améliorer la performance sur trois axes fondamentaux, souvent résumés par le triptyque \textbf{Qualité - Coûts - Délais (QCD)} :
\begin{itemize}[leftmargin=*]
    \item \textbf{Maîtriser les Coûts :} Il s'agit de traquer et d'éliminer toutes les formes de gaspillage (Muda) : surproduction, stocks excessifs, temps d'attente, transports inutiles, défauts de fabrication, mouvements superflus. Chaque ressource (humaine, matérielle, financière) doit être utilisée à son plein potentiel.
    \item \textbf{Améliorer la qualité :} L'objectif est de tendre vers le \enquote{zéro défaut}. Une qualité médiocre entraîne des coûts directs (rebuts, retouches) et indirects (insatisfaction client, perte d'image). L'optimisation vise à construire la qualité au cœur même du processus, plutôt qu'à la contrôler à la fin.
    \item \textbf{Réduire les délais :} Le \textit{Lead Time}, soit le temps total entre la commande d'un client et la livraison, est un indicateur clé de la réactivité d'une entreprise. Le réduire permet d'améliorer la satisfaction client, de diminuer les besoins en fonds de roulement (moins de stocks) et d'augmenter la flexibilité.
\end{itemize}

\subsection{Le défi de la visibilité : de la donnée à la décision}
Le principal obstacle à l'optimisation n'est souvent pas le manque de volonté, mais le manque de \textbf{visibilité}. Les processus industriels sont des systèmes complexes, avec de multiples interactions, des dépendances cachées et des flux qui traversent différents services. Les problèmes réels (les gaspillages, les goulots d'étranglement) sont souvent noyés dans cette complexité et ne sont pas immédiatement apparents.

Sans une compréhension claire et partagée de la manière dont la valeur circule (ou stagne) du début à la fin de la chaîne, les efforts d'amélioration restent locaux, parcellaires, et parfois même contre-productifs. On optimise une machine sans voir que le vrai problème est le stock qui attend en amont.

Pour optimiser, il faut d'abord \textbf{voir}. C'est précisément la raison d'être de la méthodologie VSM, qui fournit la cartographie nécessaire pour transformer la complexité en clarté.

% --- SECTION 3 : LE VALUE STREAM MAPPING COMME OUTIL DE CLARIFICATION ---
\section{Le Value Stream Mapping comme outil de clarification}

Face au défi de la visibilité évoqué précédemment, l'industrie a développé des outils pour cartographier et analyser ses processus. Parmi eux, le Value Stream Mapping (VSM), ou Cartographie des Flux de Valeur, s'est imposé comme une méthode de référence pour sa capacité à fournir une vision claire et actionnable de la chaîne de valeur.

\subsection{Définition et objectif}
Le VSM est une méthode visuelle issue du Lean Management. Son objectif principal est simple mais essentiel : \textbf{visualiser, analyser et améliorer l'ensemble des flux} (matière et information) nécessaires pour amener un produit ou un service du fournisseur jusqu'au client. Il s'agit de cartographier toutes les étapes, qu'elles ajoutent de la valeur ou non, afin d'obtenir une image complète du processus actuel. L'analyse ne se limite pas à ce qui se passe sur une machine, mais englobe également ce qui se passe \textit{entre} les machines : les temps d'attente, les stocks, les mouvements.

\subsection{L'importance du VSM dans l'industrie}
Dans un environnement économique où l'efficacité et la réactivité sont primordiales, le VSM s'avère être un \textbf{outil pertinent pour l'analyse et l'amélioration des performances}. Il permet de :
\begin{itemize}[leftmargin=*]
    \item \textbf{Identifier et quantifier les gaspillages} (Muda) : Temps d'attente, stocks excessifs, mouvements inutiles, surproduction, défauts, etc.
    \item \textbf{Localiser précisément les goulots d'étranglement} qui limitent la capacité globale du flux.
    \item \textbf{Réduire les délais de production (Lead Time)} : En optimisant le flux et en traitant les goulots, on livre plus rapidement le client.
    \item \textbf{Améliorer la productivité et réduire les coûts} : En éliminant les activités sans valeur ajoutée et en fluidifiant le processus.
    \item \textbf{Favoriser une culture d'amélioration continue (Kaizen)} : Le VSM fournit une base factuelle pour identifier les chantiers d'optimisation prioritaires.
\end{itemize}
Le VSM n'est donc pas une fin en soi, mais un \textbf{outil clé au cœur des démarches Lean}. Il sert de diagnostic initial pour comprendre où se situent les problèmes. Les informations issues de la cartographie \enquote{état actuel} permettent ensuite de concevoir un \enquote{état futur} optimisé et de définir les plans d'actions concrets pour y parvenir.

\subsection{Les acteurs concernés par la démarche}
La démarche VSM concerne de nombreux acteurs au sein d'une organisation :
\begin{itemize}[leftmargin=*]
    \item \textbf{Les ingénieurs process et méthodes} : Pour analyser et optimiser les processus de fabrication ou de service.
    \item \textbf{Les responsables production et logistique} : Pour améliorer les flux physiques et réduire les stocks.
    \item \textbf{Les responsables qualité} : Pour identifier les sources de défauts et améliorer la satisfaction client.
    \item \textbf{Les équipes d'amélioration continue / Lean managers} : Comme outil fondamental de leur méthodologie.
    \item \textbf{La direction générale} : Pour avoir une vision claire de la performance opérationnelle et orienter la stratégie.
\end{itemize}

\subsection{Les limites des méthodes traditionnelles}
Historiquement, la réalisation de VSM s'appuyait souvent sur des moyens rudimentaires qui, bien qu'utiles, présentent des freins à une analyse approfondie et dynamique :
\begin{itemize}[leftmargin=*]
    \item \textbf{Papier et Post-it} : Une excellente approche pour une première collaboration, mais difficile à maintenir, à partager à grande échelle, et impossible à analyser quantitativement de manière automatisée.
    \item \textbf{Tableurs (Excel, etc.)} : Permettent d'effectuer certains calculs, mais manquent de la visualisation intuitive qui fait la force du VSM. La représentation du flux est souvent maladroite et peu standardisée.
    \item \textbf{Logiciels de dessin génériques (Visio, Draw.io...)} : Ils offrent la flexibilité du dessin mais sont dépourvus d'intelligence métier. Ils ne comprennent pas la logique d'un VSM : il n'y a pas de calculs automatiques des indicateurs clés (temps de cycle, délai total, taux de valeur ajoutée...), et pas de liens logiques entre les éléments. La carte reste un dessin statique.
\end{itemize}
Ces méthodes traditionnelles sont souvent chronophages, sujettes aux erreurs, et ne permettent pas de simuler facilement l'impact d'améliorations potentielles. C'est précisément pour combler ce vide que la solution VSM-Tools a été conçue, en proposant un outil qui allie la clarté visuelle du VSM à la puissance de l'analyse de données.

% --- SECTION 4 : VSM-TOOLS : UNE SOLUTION DÉDIÉE À L'ANALYSE DE FLUX ---
\section{VSM-Tools : une solution dédiée à l'analyse de flux}

Face aux limites des approches traditionnelles et à l'importance reconnue du VSM, le projet VSM-Tools a été développé comme une solution logicielle intégrée, conçue pour rendre la cartographie et l'analyse des flux de valeur à la fois plus simples, plus rapides et plus intelligentes.

\subsection{Mission et vision}
\begin{itemize}[leftmargin=*]
    \item \textbf{Mission :} Rendre l'analyse VSM accessible, rapide et actionnable pour les entreprises souhaitant améliorer leur performance opérationnelle, en levant les barrières liées à la complexité des outils ou à la collecte de données.
    \item \textbf{Vision :} Devenir un outil de référence pour une cartographie VSM intuitive et efficace, qui non seulement facilite la création des cartes, mais guide également l'utilisateur dans l'analyse et l'identification des pistes d'amélioration pertinentes.
\end{itemize}

\subsection{Les apports clés de VSM-Tools}
Notre solution se distingue en apportant de la valeur sur quatre axes fondamentaux, qui répondent directement aux faiblesses des méthodes traditionnelles.

\subsubsection{Une simplification et fiabilisation du processus VSM}
L'objectif premier de VSM-Tools est de structurer et d'accélérer la phase de modélisation.
\begin{itemize}[leftmargin=*]
    \item \textbf{Modélisation guidée (Approche \enquote{Model-First}) :} Plutôt que de laisser l'utilisateur dessiner librement, l'application le guide à travers un dialogue de configuration centralisé. Cette approche garantit la cohérence logique du diagramme et standardise la saisie des informations.
    \item \textbf{Automatisation des calculs :} L'outil prend en charge les calculs clés du VSM, tels que le délai total de production (Lead Time), le temps à valeur ajoutée et le taux de valeur ajoutée. L'utilisateur obtient une analyse quantitative instantanée, sans risque d'erreur manuelle.
    \item \textbf{Gestion du cycle de vie VSM (État Actuel / État Futur) :} L'outil est conçu pour supporter le cycle complet de l'analyse. Il permet à l'utilisateur de \textbf{dupliquer facilement une cartographie \enquote{état actuel}} pour servir de base à la conception de l'\textbf{\enquote{état futur}}. L'analyste peut alors modéliser les améliorations (ex: suppression d'un stock, réduction d'un temps de cycle) et obtenir immédiatement les nouveaux indicateurs globaux, lui permettant de quantifier l'impact de ses propositions.
    \item \textbf{Gestion de projet intégrée :} Les informations du diagramme, le plan d'action et les notes associées sont stockés de manière centralisée dans une base de données locale. L'outil gère l'import et l'export au format standard \texttt{.vsmx}, assurant l'interopérabilité tout en offrant une gestion de projet robuste.
\end{itemize}

\subsubsection{Une dynamisation des données pour une VSM vivante}
Un des apports majeurs de VSM-Tools est sa capacité à se connecter à la réalité du terrain, transformant la VSM d'une photo statique en un véritable miroir de l'atelier. Pour cela, l'outil propose trois modes de collecte de données :
\begin{itemize}[leftmargin=*]
    \item \textbf{Mode Statique :} Les valeurs des indicateurs sont saisies directement dans le VSM Studio par l'analyste. Idéal pour une première cartographie.
    \item \textbf{Mode Manuel (Saisie Opérateur) :} L'application propose une interface web dédiée (développée en React) et simplifiée pour les opérateurs. Depuis leur poste, ils peuvent saisir périodiquement les données de production réelles (quantités, temps d'arrêt...).
    \item \textbf{Mode Dynamique (Connexion Directe) :} Grâce à son moteur de données (Engine) développé en Node.js, VSM-Tools peut se connecter automatiquement aux sources de données de l'entreprise via des connecteurs pour les bases de données (SQL) et les API REST.
\end{itemize}
Cette flexibilité permet à l'outil de s'adapter au niveau de maturité digitale de chaque entreprise.

\subsubsection{Une analyse augmentée par l'intelligence}
VSM-Tools ne se contente pas de représenter le flux, il aide à le comprendre.
\begin{itemize}[leftmargin=*]
    \item \textbf{Suggestions d'optimisation intelligentes :} En se basant sur les données de la carte et les principes Lean, l'outil peut mettre en évidence des goulots d'étranglement potentiels ou suggérer des pistes d'amélioration, agissant comme un assistant pour l'analyste.
    \item \textbf{Assistance par agent conversationnel :} Un chatbot, alimenté par un modèle de langage, est intégré directement dans l'application. Il peut répondre aux questions de l'utilisateur sur la méthodologie VSM, le guider dans l'utilisation du logiciel et même l'aider à interpréter certaines données.
\end{itemize}

\subsubsection{Une architecture moderne et évolutive}
La conception technique de la solution a été pensée pour être à la fois robuste et prête pour l'avenir.
\begin{itemize}[leftmargin=*]
    \item \textbf{Architecture découplée (Studio \& Engine) :} L'application est divisée en une interface utilisateur de bureau pour la modélisation (\textbf{Studio}, en Electron) et un moteur de données en arrière-plan (\textbf{Engine}, en Node.js). Cette séparation des préoccupations rend le système plus maintenable et performant.
    \item \textbf{Écosystème d'interfaces :} En plus du Studio, l'architecture supporte des interfaces web dédiées (développées en React) pour les cas d'usage spécifiques que sont la \textbf{saisie opérateur} et la \textbf{visualisation de dashboards} en lecture seule.
\end{itemize}

En résumé, VSM-Tools est conçu pour être un outil complet, qui accompagne l'utilisateur depuis la collecte de données jusqu'à l'identification d'actions d'amélioration, en s'appuyant sur une interface guidée, des calculs automatisés et des fonctionnalités intelligentes.

% --- SECTION 5 : ARCHITECTURE TECHNIQUE ---
\section{Architecture technique}

Pour répondre aux exigences fonctionnelles du projet, une architecture logicielle moderne, modulaire et découplée a été conçue et implémentée. Le choix s'est porté sur un écosystème basé sur JavaScript/TypeScript, en tirant parti des frameworks Electron pour l'application de bureau, Node.js pour le moteur de services, et React pour les interfaces web.

\subsection{Vue d'ensemble de l'architecture}
L'architecture de VSM-Tools est organisée en trois composants principaux qui communiquent entre eux via des API internes. Cette séparation garantit la modularité, la performance et la maintenabilité de la solution.

\begin{figure}[H]
    \centering
    % Placeholder pour un schéma d'architecture.
    \fbox{
        \begin{minipage}{0.8\textwidth}
            \textbf{Schéma d'Architecture VSM-Tools}
            \vspace{1cm}
            \begin{itemize}
                \item \textbf{VSM Studio (Electron App)} : Interface de modélisation principale.
                      \begin{itemize}
                          \item Processus Principal (Node.js)
                          \item Processus de Rendu (Chromium - HTML/CSS/JS)
                      \end{itemize}
                \item \textbf{VSM Engine (Node.js Service)} : Moteur de données et de connectivité.
                      \begin{itemize}
                          \item API REST interne
                          \item Logique métier, calculs, connecteurs
                      \end{itemize}
                \item \textbf{Base de Données PostgreSQL} : Stockage des projets.
                \item \textbf{Interfaces Web (React Apps)} :
                      \begin{itemize}
                          \item Dashboard de visualisation
                          \item Interface de saisie opérateur
                      \end{itemize}
            \end{itemize}
        \end{minipage}
    }
    \caption{Diagramme de l'architecture générale de VSM-Tools.}
    \label{fig:architecture}
\end{figure}

\subsection{Le VSM Studio : l'application de bureau}
Le VSM Studio est l'application principale de configuration et de modélisation. La technologie \textbf{Electron} a été choisie car elle permet de créer des applications de bureau multi-plateformes (Windows, macOS, Linux) en utilisant des technologies web standard (HTML, CSS, JavaScript).

Electron fonctionne avec deux types de processus :
\begin{itemize}[leftmargin=*]
    \item \textbf{Le Processus principal :} Il s'exécute en arrière-plan et a un accès complet à l'environnement Node.js. Il est responsable de la gestion des fenêtres de l'application, des menus natifs du système d'exploitation, et de la communication avec le VSM Engine et la base de données. C'est le chef d'orchestre de l'application.
    \item \textbf{Le Processus de rendu :} Chaque fenêtre de l'application est un processus de rendu indépendant. Il s'agit essentiellement d'une fenêtre de navigateur web dans laquelle notre interface utilisateur, développée avec des frameworks JavaScript modernes, est affichée. Il n'a pas d'accès direct aux ressources du système et doit communiquer avec le Processus Principal pour effectuer des opérations sensibles.
\end{itemize}

\subsection{Le VSM Engine : le moteur de services}
Le VSM Engine est le cœur logique de l'application. Il fonctionne comme un service en arrière-plan, lancé et géré par le Processus Principal d'Electron. Le choix de \textbf{Node.js} permet de capitaliser sur un écosystème JavaScript unifié et performant pour les opérations asynchrones.

Les responsabilités de l'Engine sont les suivantes :
\begin{itemize}[leftmargin=*]
    \item \textbf{Logique métier :} Il contient toute la logique de validation du modèle de données VSM, les calculs automatiques (Lead Time, \%VA), et l'implémentation des suggestions d'optimisation.
    \item \textbf{Gestion de la base de données :} Il assure la communication avec le serveur de base de données central \textbf{PostgreSQL} pour sauvegarder et récupérer les données des projets (diagrammes, notes, plans d'action).
    \item \textbf{Connectivité externe :} Il gère les connecteurs pour le mode \enquote{Dynamique}. C'est lui qui exécute les requêtes vers les bases de données externes ou les API REST pour récupérer les valeurs des indicateurs.
    \item \textbf{Exposition des données via API :} L'Engine expose une API REST interne (via un framework comme Express.js) que les différents fronts (le VSM Studio et les interfaces web) peuvent consommer pour obtenir les données des VSM.
\end{itemize}

\subsection{La base de données centralisée }
La persistance des données du projet VSM-Tools ne s'appuie pas sur de simples fichiers locaux, mais sur un système de gestion de base de données relationnelle (SGBDR) robuste et éprouvé : \textbf{PostgreSQL}.

Ce choix architectural est structurant et apporte plusieurs avantages significatifs par rapport à une solution de stockage locale :
\begin{itemize}[leftmargin=*]
    \item \textbf{Centralisation et partage des données :} Toutes les données des projets (diagrammes, notes, plans d'action) sont stockées de manière centralisée sur un serveur de base de données. Cela signifie que plusieurs utilisateurs du VSM Studio (s'ils sont connectés au même serveur) peuvent potentiellement accéder et travailler sur le même ensemble de projets.
    \item \textbf{Robustesse et intégrité :} PostgreSQL est reconnu pour sa fiabilité, sa gestion avancée des transactions (ACID) et ses mécanismes de contraintes d'intégrité. Cela garantit que les données du modèle VSM sont stockées de manière sûre et cohérente.
    \item \textbf{Performance et scalabilité :} L'utilisation d'un SGBDR performant permet de gérer efficacement un grand volume de données et de projets. Le serveur de base de données peut être dimensionné indépendamment de l'application cliente.
    \item \textbf{Accès unifié :} Le \textbf{VSM Engine} est le seul composant de notre architecture qui communique directement avec la base de données PostgreSQL. Il agit comme une couche d'abstraction (ou "data access layer"). Le VSM Studio et les interfaces web n'accèdent jamais directement à la base de données ; ils passent obligatoirement par l'API de l'Engine. Cette centralisation de l'accès sécurise les données et simplifie la maintenance.
\end{itemize}

L'utilisation de PostgreSQL positionne VSM-Tools non pas comme un simple outil de bureau isolé, mais comme une plateforme prête pour des cas d'usage collaboratifs et d'entreprise, où les données VSM peuvent être gérées de manière centralisée et sécurisée.

\subsection{Les interfaces web}
Pour les besoins de visualisation et de saisie de données en dehors du VSM Studio, des applications web légères ont été développées avec la bibliothèque \textbf{React}.
\begin{itemize}[leftmargin=*]
    \item \textbf{L'Interface de saisie opérateur :} Une interface ultra-simplifiée, accessible via un navigateur web, conçue pour être utilisée sur des postes d'atelier. Elle ne permet que de sélectionner une étape de processus et de saisir les valeurs pour les indicateurs qui lui sont assignés.
    \item \textbf{Le Dashboard de visualisation :} Une interface en lecture seule, également accessible via un navigateur, qui permet de consulter les diagrammes VSM et leurs indicateurs à jour. C'est l'outil idéal pour le management et le partage de l'information.
\end{itemize}
Ces applications web sont servies par le VSM Engine (Node.js) et consomment la même API REST que le VSM Studio, garantissant la cohérence des données affichées.

\subsection{L'intégration de l'intelligence artificielle}
L'agent conversationnel est intégré directement dans l'interface du VSM Studio. Pour assurer la sécurité et la performance, la communication avec l'API du modèle de langage (Large Language Model - LLM) est gérée par le Processus Principal d'Electron. Celui-ci est responsable de :
\begin{itemize}[leftmargin=*]
    \item \textbf{Sécuriser les clés d'API :} Les clés d'authentification ne sont jamais exposées côté client (dans le processus de rendu).
    \item \textbf{Orchestrer les interactions :} Il gère le cycle complet de la conversation, y compris les mécanismes avancés comme le "tool calling" (la capacité pour le modèle de demander l'exécution d'une fonction locale, comme "ajouter une étape de processus de production"), ce qui ouvre la voie à des interactions plus riches à l'avenir.
\end{itemize}

% --- SECTION 6 : LE PARCOURS UTILISATEUR ---
\section{Le parcours utilisateur}

La solution VSM-Tools a été conçue pour s'adresser à différents acteurs de l'entreprise, chacun ayant des besoins et des modes d'interaction spécifiques. L'architecture de la solution se reflète dans trois parcours utilisateurs distincts mais interconnectés, qui s'articulent tous autour de la même base de données centrale.

\subsection{Le parcours de l'analyste : conception et analyse}
L'analyste ou l'ingénieur processus est l'utilisateur principal du \textbf{VSM Studio}, l'application de bureau. Son parcours est le plus complet, car il est responsable de la création, de la configuration et de l'analyse approfondie de la VSM.

\begin{enumerate}[leftmargin=*]
    \item \textbf{Initialisation du projet :} L'analyste commence par créer un nouveau projet, qui est une entrée dédiée dans la base de données PostgreSQL. Il peut également importer un fichier \texttt{.vsmx} existant pour initialiser un projet.

    \item \textbf{Configuration du modèle de données :} L'étape centrale du travail se déroule dans le \textbf{Dialogue de Configuration Central}. L'analyste ne dessine pas, il modélise en remplissant des formulaires structurés pour :
          \begin{itemize}
              \item Définir les sources de données externes (connexions aux bases de données, API REST...).
              \item Créer les entités du flux : étapes de processus, fournisseurs, clients, etc.
              \item Définir la séquence logique du flux principal, en ordonnançant les étapes et les éléments intermédiaires (stocks, flux poussés).
              \item Attacher des \textbf{indicateurs} à chaque élément pertinent et choisir leur mode de collecte de données : \textit{statique} (valeur fixe), \textit{manuel} (saisie par un opérateur), ou \textit{dynamique} (collecte automatique).
          \end{itemize}

    \item \textbf{Visualisation et analyse de l'état actuel :} Une fois la configuration appliquée, l'algorithme de layout génère automatiquement le diagramme VSM sur le canevas. L'analyste peut alors :
          \begin{itemize}
              \item Naviguer dans le diagramme et analyser les indicateurs globaux calculés automatiquement (Lead Time, taux de valeur ajoutée...).
              \item Définir des \textbf{règles de détection intelligentes} pour que l'outil identifie et surligne automatiquement les goulots d'étranglement, les gaspillages (ex: stocks trop élevés) et les opportunités d'amélioration sur la base de seuils ou de logiques prédéfinies.
              \item Interagir avec l'\textbf{agent conversationnel} (IA) pour obtenir de l'aide, poser des questions sur la méthodologie ou demander des suggestions d'optimisation.
          \end{itemize}

    \item \textbf{Conception de l'état futur :} Une fois l'état actuel analysé et les goulots identifiés, l'analyste passe à la conception de la solution. Pour cela, il utilise la fonctionnalité de l'outil pour \textbf{dupliquer son diagramme} afin de créer une version \enquote{état futur}. Sur cette nouvelle carte, il modélise les améliorations envisagées : il supprime un stock, modifie un flux pour le passer en "tiré", réduit un temps de changement de série... L'outil met alors à jour instantanément les indicateurs globaux (Lead Time, \%VA), ce qui permet de \textbf{mesurer et de valider l'efficacité} des améliorations proposées avant même leur déploiement sur le terrain.

    \item \textbf{Enrichissement et plan d'action :} L'analyse débouche sur des actions concrètes. L'analyste peut :
          \begin{itemize}
              \item Ajouter des \textbf{points d'amélioration (Kaizen)} directement sur le diagramme pour marquer les zones à traiter.
              \item Rédiger des \textbf{notes} et définir un \textbf{plan d'action} structuré, ces informations étant directement stockées dans la base de données du projet.
              \item Exporter le diagramme finalisé au format \texttt{.vsmx} pour l'archivage ou le partage.
          \end{itemize}
\end{enumerate}

\subsubsection{Focus sur la configuration d'un indicateur dynamique}
La puissance de VSM-Tools réside dans sa capacité à connecter les indicateurs à des données réelles. Le processus de configuration de ce mode dynamique a été conçu pour être à la fois robuste et intuitif, en deux étapes clés :

\begin{enumerate}[leftmargin=*]
    \item \textbf{Étape 1 : Définition des sources de données.}
          Dans une section dédiée du Dialogue de Configuration, l'analyste crée une bibliothèque de toutes les connexions de données disponibles.
          \begin{itemize}
              \item Pour une \textbf{Base de données SQL}, il renseigne les informations de connexion (Nom, Hôte, Port, Nom de la base, Utilisateur, Mot de passe), comme illustré par l'interface de création de source.
              \item Pour une \textbf{API REST}, il spécifie l'URL de base et le type d'authentification requis (Bearer Token, API Key, etc.).
          \end{itemize}
          Cette étape centralise la gestion des connexions, permettant de les réutiliser pour plusieurs indicateurs sans avoir à ressaisir les informations.

    \item \textbf{Étape 2 : Liaison et configuration de l'indicateur.}
          Lors de l'édition d'un indicateur pour une étape de processus (par exemple, le \enquote{Temps de Cycle}), l'analyste choisit le mode \textbf{\enquote{Dynamique}}.
          \begin{itemize}
              \item Un panneau de configuration apparaît. L'analyste sélectionne d'abord dans un menu déroulant la \textbf{source de données} qu'il a préalablement définie.
              \item En fonction du type de la source choisie, un champ contextuel final apparaît pour spécifier comment récupérer la donnée :
                    \begin{itemize}
                        \item Si la source est de type SQL, un champ \textbf{\enquote{Requête SQL}} lui permet de saisir la requête exacte (ex: \texttt{SELECT cycle\_time FROM stats WHERE machine\_id = 'SHAPING'}).
                        \item Si la source est de type API REST, un champ \textbf{\enquote{Endpoint}} lui permet de spécifier le chemin de la ressource (ex: \texttt{/machines/shaping/cycletime}).
                    \end{itemize}
          \end{itemize}
\end{enumerate}
Ce processus en deux temps garantit une configuration claire et sécurisée, tout en offrant une flexibilité maximale pour interroger précisément la donnée nécessaire.

\subsection{Le parcours de l'opérateur : la collecte de données terrain}
Le parcours de l'opérateur est conçu pour être aussi simple et rapide que possible, afin de minimiser l'impact sur ses tâches de production.

\begin{enumerate}[leftmargin=*]
    \item \textbf{Accès :} Depuis un poste de travail dans l'atelier, l'opérateur se connecte via un navigateur web à une \textbf{interface de saisie dédiée}.
    \item \textbf{Saisie :} L'interface lui présente une liste claire des indicateurs qu'il doit renseigner (ceux configurés en mode \textit{manuel}). Il sélectionne son poste ou son processus, entre les valeurs demandées (ex: quantités produites, temps d'arrêt) et valide.
    \item \textbf{Synchronisation :} Les données sont instantanément envoyées au VSM Engine et mises à jour dans la base de données PostgreSQL, rendant l'information disponible pour l'analyste et le manager.
\end{enumerate}

\subsection{Le parcours du manager : la supervision et la prise de décision}
Le manager ou le chef d'équipe a besoin d'une vision synthétique et à jour de la performance du flux.

\begin{enumerate}[leftmargin=*]
    \item \textbf{Accès :} Depuis son ordinateur ou en salle de réunion, il accède via un navigateur web au \textbf{dashboard de visualisation}.
    \item \textbf{Consultation :} Il visualise le diagramme VSM en mode lecture seule. Les indicateurs affichés sur la carte se rafraîchissent en temps réel (ou à intervalles réguliers), consolidant les données issues des connexions dynamiques et des saisies des opérateurs.
    \item \textbf{Prise de décision :} Grâce à cette vue d'ensemble toujours à jour, et aux alertes visuelles générées par les règles de détection (ex: un goulot d'étranglement surligné en rouge), il peut suivre la performance, identifier les dérives et prendre des décisions éclairées pour piloter l'activité.
\end{enumerate}

Ce fonctionnement en trois parcours distincts mais intégrés permet à VSM-Tools de s'insérer de manière cohérente dans l'organisation de l'entreprise, en fournissant le bon outil, avec le bon niveau d'information, à la bonne personne.

% --- SECTION 7 : ANALYSE CONCURRENTIELLE ET POSITIONNEMENT ---
\section{Analyse concurrentielle et positionnement}

Pour positionner VSM-Tools de manière pertinente, il est essentiel de comprendre le paysage des solutions existantes. Le marché de la cartographie des flux de valeur est segmenté en plusieurs catégories d'outils, chacune avec ses forces et ses faiblesses. Cette analyse nous a permis de dégager clairement les axes de différenciation de notre solution.

\subsection{Panorama des solutions existantes}
Nous pouvons classer les outils couramment utilisés en trois grandes familles :

\begin{enumerate}[leftmargin=*]
    \item \textbf{Les outils de dessin génériques :} Très accessibles, ces outils permettent de dessiner n'importe quel type de diagramme.
          \begin{itemize}
              \item \textit{Exemples :} Microsoft Visio, Lucidchart, Draw.io, SmartDraw.
              \item \textit{Forces :} Flexibilité, faible coût, prise en main rapide.
              \item \textit{Limites pour le VSM :} Ils n'ont aucune intelligence métier. La modélisation est entièrement manuelle, il n'y a aucun calcul automatique des indicateurs Lean, et aucune connexion de données dynamique. Ils produisent des dessins statiques, pas des outils d'analyse.
          \end{itemize}

    \item \textbf{Les outils VSM spécialisés :} Conçus spécifiquement pour le VSM et l'analyse de flux, ils intègrent une logique métier.
          \begin{itemize}
              \item \textit{Exemples :} visTABLE\textregistered{}, Simcad Pro.
              \item \textit{Forces :} Calculs intégrés, bibliothèques de symboles VSM, et souvent, des capacités de simulation avancées.
              \item \textit{Limites :} Peuvent être complexes à prendre en main, avec des coûts de licence élevés. Leur focus est souvent sur la simulation d'événements discrets ou l'aménagement d'usine, ce qui peut être surdimensionné pour une analyse VSM centrée sur les flux.
          \end{itemize}

    \item \textbf{Les plateformes de Value Stream Management (VSMgt) :} Une catégorie plus récente et de plus haut niveau, souvent orientée vers la gestion de flux de valeur dans le développement logiciel ou les grands portefeuilles de produits.
          \begin{itemize}
              \item \textit{Exemples :} Planview, Axify.
              \item \textit{Forces :} Intégration profonde avec les systèmes d'entreprise (comme Jira, Git), analyse de données à grande échelle et utilisation de l'IA pour générer des aperçus.
              \item \textit{Limites :} Ce sont des solutions d'entreprise très coûteuses et complexes à déployer. Elles sont souvent trop abstraites pour une analyse VSM terrain focalisée sur un flux de production physique.
          \end{itemize}
\end{enumerate}

\subsection{Comparaison avec les acteurs clés}

\paragraph{visTABLE\textregistered{}} est un outil puissant, mais son principal domaine d'excellence est l'aménagement d'usine (Factory Layout). Il est excellent pour visualiser les flux de matière dans un espace 2D/3D et optimiser les distances de transport. Cependant, son approche est moins centrée sur les flux d'information et les indicateurs Lean globaux que ne l'est une démarche VSM pure.

\paragraph{Simcad Pro} est une plateforme de simulation d'événements discrets extrêmement performante. Elle permet de construire des modèles dynamiques très détaillés d'une usine pour tester des scénarios. Sa complexité et son coût la réservent à des experts de la simulation. Pour une équipe d'amélioration continue qui a besoin de réaliser une VSM rapidement, l'outil peut être intimidant et surdimensionné.

\paragraph{Lucidchart} représente le meilleur des outils de dessin génériques. Il est collaboratif, basé sur le web et propose des modèles VSM. Cependant, il reste fondamentalement un outil de dessin. Les calculs de la timeline doivent être faits manuellement, et il n'y a pas de connexion native aux données de production pour rendre la carte vivante.

\subsection{Les atouts différenciants de VSM-Tools}
C’est dans cet écosystème que VSM-Tools apporte sa valeur ajoutée unique, en créant un pont entre la simplicité des outils de dessin et la puissance des outils spécialisés, tout en y ajoutant une couche d'intelligence et de flexibilité inédite.

\begin{itemize}[leftmargin=*]
    \item \textbf{Approche \enquote{Model-First} unique :} Contrairement aux outils de dessin, VSM-Tools guide l'utilisateur dans la construction d'un modèle de données cohérent. Cela garantit la fiabilité de l'analyse et produit automatiquement un diagramme professionnel. La qualité du résultat ne dépend plus du talent de dessinateur de l'utilisateur.

    \item \textbf{Flexibilité inégalée de la collecte de données :} C'est l'atout majeur de notre solution. En proposant trois modes (statique, saisie opérateur manuelle, et dynamique via connecteurs), VSM-Tools s'adapte à la maturité digitale de l'entreprise. Une équipe peut commencer avec une VSM statique, puis la faire évoluer progressivement vers un tableau de bord dynamique sans changer d'outil.

    \item \textbf{Intelligence accessible :} VSM-Tools intègre des fonctionnalités intelligentes (suggestions d'optimisation, agent conversationnel) d'une manière qui vise à assister l'utilisateur, et non à le submerger de complexité comme peuvent le faire les grandes plateformes de simulation.

    \item \textbf{Architecture ouverte et moderne :} En s'appuyant sur des technologies modernes et répandues (Electron, Node.js, React), VSM-Tools est nativement multi-plateformes et plus facile à faire évoluer et à intégrer que des solutions propriétaires plus anciennes.
\end{itemize}

En conclusion, VSM-Tools ne cherche pas à être l'outil de simulation le plus complexe ni la plateforme de management la plus large. Il se positionne comme la solution \textbf{la plus pragmatique et la plus adaptable} pour les équipes d'amélioration continue qui veulent réaliser des VSM fiables, basées sur des données réelles, et en tirer des plans d'action concrets.

% --- SECTION 8 : POSITIONNEMENT DANS L'ÉCOSYSTÈME INDUSTRIEL ---
\section{Positionnement dans l'écosystème industriel}

\subsection{La pyramide d'automatisation comme cadre de référence}
Pour comprendre où VSM-Tools apporte sa valeur, il est utile de le situer par rapport au modèle classique de la pyramide d'automatisation, qui hiérarchise les systèmes d'information et de contrôle d'une entreprise industrielle :
\begin{itemize}[leftmargin=*]
    \item \textbf{Niveau stratégique (sommet)} : Systèmes de planification des ressources de l'entreprise (ERP - Enterprise Resource Planning) pour la gestion globale (finances, ventes, achats, planification long terme).
    \item \textbf{Niveau pilotage / exécution} : Systèmes de gestion de la production (MES - Manufacturing Execution System) pour le suivi en temps réel des opérations, la gestion des ordres de fabrication, la traçabilité.
    \item \textbf{Niveau supervision} : Systèmes de supervision et d'acquisition de données (SCADA - Supervisory Control And Data Acquisition) pour le contrôle et la visualisation des processus industriels.
    \item \textbf{Niveau contrôle} : Automates programmables (PLC - Programmable Logic Controller) qui commandent directement les machines.
    \item \textbf{Niveau terrain (base)} : Capteurs et actionneurs sur les équipements physiques.
\end{itemize}

\begin{figure}[H]
    \centering
    \includegraphics[width=0.7\textwidth]{../../../images/automation_pyramide.png}
    \caption{La pyramide d'automatisation.}
    \label{fig:pyramide_auto}
\end{figure}

\subsection{La place de VSM-Tools dans cet écosystème}
VSM-Tools ne se situe pas directement \textit{dans} une couche unique de cette pyramide opérationnelle. Il se positionne plutôt \textbf{en complément, comme une couche d'analyse et d'aide à la décision qui interagit avec plusieurs niveaux}.

Son rôle principal est d'\textbf{optimiser les flux de valeur} qui sont \textit{gérés et exécutés} par les systèmes comme l'ERP et le MES. VSM-Tools est un pont analytique :
\begin{itemize}[leftmargin=*]
    \item Il \textbf{consomme} des informations issues des niveaux supérieurs (planification de l'ERP) et intermédiaires (données réelles de production du MES).
    \item Il \textbf{produit} des analyses, des objectifs d'amélioration (réduction des délais, optimisation des stocks) et des règles de détection qui vont influencer la stratégie (niveau ERP) et guider les actions d'optimisation au niveau de l'exécution (niveau MES).
\end{itemize}
VSM-Tools est donc un \textbf{outil focalisé sur la compréhension et l'amélioration des processus transversaux}, là où les autres systèmes sont souvent centrés sur la gestion transactionnelle (ERP) ou le contrôle de l'exécution (MES/SCADA).

\subsection{Synergies et interactions}
La véritable force de VSM-Tools est décuplée par ses interactions avec les autres systèmes de l'écosystème.

\begin{itemize}[leftmargin=*]
    \item \textbf{Alimentation en données fiables} : Grâce à son moteur et ses connecteurs, VSM-Tools est conçu pour récupérer des données agrégées et fiabilisées directement depuis le MES (temps de cycle moyens, TRS, temps d'arrêt...) ou l'ERP (coûts standards, niveaux de stock...). Cela rend la création de la VSM \enquote{état actuel} plus rapide, plus précise, et permet de construire une cartographie réellement \enquote{vivante}.

    \item \textbf{Orientation des actions et pilotage affiné} : L'analyse réalisée dans VSM-Tools (identification des goulots, définition de règles de détection) permet de fixer des objectifs plus robustes et précis (par exemple, objectifs de temps de cycle, niveaux de stock cibles...). Ces objectifs peuvent ensuite être utilisés pour paramétrer ou piloter plus finement les systèmes MES et ERP, assurant un meilleur alignement entre l'analyse stratégique des flux et l'exécution opérationnelle.

    \item \textbf{Complémentarité fondamentale} : Il ne s'agit pas de remplacer les systèmes existants. La répartition des rôles est claire :
          \begin{itemize}
              \item L'\textbf{ERP} gère les ressources globalement.
              \item Le \textbf{MES} exécute et contrôle les opérations au quotidien.
              \item \textbf{VSM-Tools} sert à \textit{comprendre, analyser en profondeur et concevoir} l'amélioration des flux. Il apporte la vision \enquote{flux} dynamique et l'intelligence analytique qui manquent souvent aux systèmes transactionnels.
          \end{itemize}
\end{itemize}
En intégrant VSM-Tools dans l'écosystème digital de l'entreprise, on crée une boucle vertueuse où l'analyse avancée des flux guide l'action opérationnelle, et les données opérationnelles nourrissent cette analyse pour une amélioration continue et informée.

% --- SECTION 9 : DÉFIS ET LIMITES ACTUELLES ---
\section{Défis et limites actuelles}

Dans une démarche de transparence et d'analyse critique, il est essentiel de reconnaître les défis inhérents au projet et les limites de la version actuelle de VSM-Tools. Ces points ne diminuent pas la valeur de la solution, mais définissent un cadre réaliste pour son utilisation et tracent la voie pour ses futures évolutions.

\subsection{La dépendance à la qualité des données d'entrée}
Comme tout outil d'analyse, et \textit{a fortiori} pour ses fonctions d'analyse intelligente, la pertinence des résultats fournis par VSM-Tools dépend directement de la qualité, de la complétude et de la fiabilité des données qui l'alimentent. Des données erronées ou incomplètes mèneront inévitablement à une analyse faussée et à des suggestions potentiellement contre-productives. La collecte rigoureuse des données, que ce soit par connexion dynamique, saisie opérateur ou saisie statique, reste donc une étape préalable essentielle à la charge de l'utilisateur.

\subsection{Un outil au service d'une démarche méthodologique}
VSM-Tools est un puissant facilitateur, mais il ne remplace pas la compréhension et l'application de la méthodologie Lean et VSM. Pour tirer pleinement parti de l'outil, notamment de ses suggestions ou de la détection des goulots d'étranglement, les utilisateurs doivent comprendre les principes sous-jacents. L'outil maximise son impact lorsqu'il est intégré dans une culture d'amélioration continue et piloté par des équipes formées à la démarche.

\subsection{La gestion de la sécurité et des droits d'accès}
Pour offrir une expérience centralisée et collaborative, VSM-Tools s'appuie sur une architecture où les données sont stockées sur un serveur. Cette approche puissante introduit des défis en matière de sécurité et de gestion des utilisateurs qui doivent être adressés dans un déploiement en production.
\begin{itemize}[leftmargin=*]
    \item \textbf{Authentification :} Dans la version actuelle, le mécanisme d'authentification des utilisateurs (login/mot de passe) n'a pas été implémenté. Pour une utilisation en entreprise, il sera indispensable d'intégrer un système d'authentification robuste pour sécuriser l'accès à la plateforme.
    \item \textbf{Gestion des droits (ACL) :} Actuellement, un utilisateur authentifié aurait accès à l'ensemble des projets. Un système de contrôle d'accès (Access Control List - ACL) serait nécessaire pour gérer des droits fins (qui peut voir, modifier, ou administrer un projet VSM).
\end{itemize}
Ces aspects de sécurité sont des prérequis essentiels pour un déploiement à grande échelle.

\subsection{Le périmètre fonctionnel de la version actuelle}
Le développement a été priorisé pour construire un socle solide et des fonctionnalités à forte valeur ajoutée. Par conséquent, le périmètre de la version actuelle, bien que fonctionnel, présente certaines limites et contraintes de conception par rapport à une vision à long terme :
\begin{itemize}[leftmargin=*]
    \item \textbf{Contraintes de modélisation du flux :} Pour simplifier la logique de l'algorithme de layout et la gestion du flux principal dans cette première version, une contrainte a été posée : un diagramme VSM ne peut comporter qu'\textbf{un seul acteur Fournisseur} et \textbf{un seul acteur Client}. La gestion de chaînes de valeur plus complexes avec de multiples fournisseurs ou points de distribution est un axe d'évolution futur.

    \item \textbf{Absence de simulation et de prédiction :} Les fonctionnalités d'analyse avancée comme la simulation de scénarios \enquote{what-if} ou l'analyse prédictive des goulots futurs n'ont pas été implémentées.

    \item \textbf{Connectivité système générique vs. spécifique :} Le mode \enquote{Dynamique} est entièrement fonctionnel et permet de se connecter à n'importe quelle source de données exposant une base de données \textbf{SQL} ou une \textbf{API REST}. Cependant, la version actuelle ne dispose pas d'une bibliothèque de connecteurs \enquote{clés en main} pour des logiciels métiers spécifiques (ex: un connecteur "SAP"). La configuration d'une connexion dynamique nécessite donc une connaissance technique de la requête SQL ou de l'endpoint de l'API à interroger.

    \item \textbf{Fonctionnalités collaboratives avancées :} L'architecture centralisée pose les bases de la collaboration. Cependant, des fonctionnalités de travail collaboratif en temps réel sur une même cartographie ne sont pas implémentées. La collaboration reste asynchrone.
\end{itemize}
Ces limites sont identifiées et constituent des axes clairs et prioritaires pour les évolutions futures de la solution.

% --- SECTION 10 : PERSPECTIVES ET ÉVOLUTIONS FUTURES ---
\section{Perspectives et évolutions futures}

La version actuelle de VSM-Tools constitue un socle fonctionnel robuste. Cependant, le potentiel de la solution est bien plus vaste. Les limites identifiées dans la section précédente dessinent naturellement la feuille de route pour les évolutions futures. L'objectif est de transformer VSM-Tools en une plateforme encore plus intelligente, collaborative et intégrée.

\subsection{L'enrichissement des capacités d'analyse}
C'est l'axe d'évolution le plus stratégique pour augmenter la valeur ajoutée de l'outil.
\begin{itemize}[leftmargin=*]
    \item \textbf{Implémentation de la simulation \enquote{What-if} :} Permettre aux utilisateurs de dupliquer un état actuel, de modifier des paramètres (temps de cycle, nombre d'opérateurs, taux de rebut...) et de simuler l'impact de ces changements sur les indicateurs de performance globaux (Lead Time, coût, capacité).
    \item \textbf{Développement de l'analyse prédictive :} Intégrer des modèles statistiques ou de machine learning simples pour anticiper les futurs goulots d'étranglement en fonction des variations de la demande client ou de la performance des équipements.
    \item \textbf{Tableau de bord de comparaison (Actuel vs. Futur) :} Développer une vue dédiée qui présenterait un tableau de bord comparatif des indicateurs clés entre les deux états, avec des calculs de gains automatisés (ex: "-2 jours de Lead Time", "+15\% de valeur ajoutée"), pour rendre la communication des résultats encore plus percutante.
\end{itemize}

\subsection{Le renforcement de la connectivité et des intégrations}
Pour réduire l'effort de configuration manuelle et garantir des données toujours à jour, une priorité sera de développer des intégrations plus poussées.
\begin{itemize}[leftmargin=*]
    \item \textbf{Bibliothèque de connecteurs métiers :} Développer une bibliothèque de connecteurs pré-configurés pour les principaux systèmes du marché (ERP comme SAP, MES comme Wonderware...). L'utilisateur n'aurait qu'à sélectionner le système et fournir ses identifiants, au lieu d'écrire des requêtes SQL ou des chemins d'API manuellement.
\end{itemize}

\subsection{L'extension de la modélisation et de la collaboration}
Pour adresser des cas d'usage plus complexes et améliorer le travail d'équipe.
\begin{itemize}[leftmargin=*]
    \item \textbf{Gestion des flux complexes :} Faire évoluer le modèle de données et l'algorithme de layout pour supporter des chaînes de valeur avec \textbf{plusieurs fournisseurs et/ou plusieurs clients}, ainsi que des flux avec des branches (divergence/convergence).
    \item \textbf{Fonctionnalités collaboratives temps réel :} Tirer parti de l'architecture centralisée pour implémenter des fonctionnalités de collaboration en direct, permettant à plusieurs utilisateurs de visualiser les modifications sur une même cartographie simultanément.
    \item \textbf{Gestion avancée des droits et des versions :} Mettre en place un système de contrôle d'accès (ACL) pour définir des rôles (lecteur, éditeur, administrateur) sur les projets, ainsi qu'un historique des versions pour pouvoir comparer les VSM à différents moments.
\end{itemize}

\subsection{L'évolution du modèle de déploiement}
Pour répondre aux différents besoins des entreprises et simplifier l'accès à la solution.
\begin{itemize}[leftmargin=*]
    \item \textbf{Offre Cloud / SaaS (Software as a Service) :} Proposer une version entièrement hébergée de VSM-Tools. Cela éliminerait toute contrainte d'installation et de maintenance pour le client, permettrait un accès via un simple navigateur web, et faciliterait les mises à jour. Ce modèle ouvrirait également la voie à des modèles d'abonnement flexibles.
\end{itemize}

Ces évolutions visent à faire de VSM-Tools un partenaire encore plus stratégique dans la démarche d'excellence opérationnelle des entreprises, en le rendant plus prédictif, plus intégré et plus collaboratif.

% --- SECTION 11 : MODÈLE ÉCONOMIQUE ---
\section{Modèle économique}

Le développement de VSM-Tools a été mené non seulement comme un projet technique, mais aussi comme une réflexion entrepreneuriale. Pour assurer la pérennité et la croissance de la solution, un modèle économique clair a été envisagé. L'objectif est de proposer une offre qui soit à la fois accessible pour permettre une large adoption et suffisamment valorisée pour financer les développements futurs.

\subsection{Segments de clientèle cibles}
L'analyse du marché nous a permis d'identifier trois segments de clientèle principaux, chacun avec des besoins et des moyens différents :
\begin{enumerate}[leftmargin=*]
    \item \textbf{Les Individuels et le Milieu Académique :} Étudiants, enseignants, chercheurs et consultants indépendants qui ont besoin d'un outil VSM puissant pour des projets ponctuels ou à des fins pédagogiques.
    \item \textbf{Les PME et les Équipes d'Amélioration Continue :} Petites et moyennes entreprises ou des départements spécifiques au sein de grands groupes qui cherchent à digitaliser et à dynamiser leur démarche VSM de manière pragmatique et à un coût maîtrisé.
    \item \textbf{Les Grandes Entreprises :} Organisations de grande taille qui ont des besoins avancés en matière d'intégration, de sécurité, de collaboration et de support.
\end{enumerate}

\subsection{Stratégie de tarification : un modèle \textit{Freemium} à plusieurs niveaux}
Pour adresser ces différents segments, nous avons opté pour une stratégie de tarification à trois niveaux. Cette approche permet de proposer une porte d'entrée gratuite pour découvrir l'outil, tout en offrant des fonctionnalités de plus en plus puissantes pour les utilisateurs professionnels et les entreprises.

\subsubsection{Niveau 1 : VSM-Tools \textit{Community} (Gratuit)}
\begin{description}[leftmargin=*, style=unboxed]
    \item[Objectif :] Encourager l'adoption, la formation et l'utilisation individuelle. Servir de produit d'appel.
    \item[Public Cible :] Étudiants, enseignants, consultants indépendants.
    \item[Fonctionnalités Incluses :]
          \begin{itemize}
              \item Création d'un nombre limité de projets (ex: 3 projets maximum).
              \item Ensemble complet des fonctionnalités de modélisation.
              \item Génération automatique du diagramme et des calculs de base.
              \item Mode de collecte de données \textbf{statique} uniquement.
              \item Export des diagrammes (potentiellement avec un filigrane).
          \end{itemize}
\end{description}

\subsubsection{Niveau 2 : VSM-Tools \textit{Professional} (Abonnement payant)}
\begin{description}[leftmargin=*, style=unboxed]
    \item[Objectif :] Fournir la pleine puissance de l'outil aux professionnels et aux équipes qui souhaitent rendre leurs VSM vivantes.
    \item[Public Cible :] PME, équipes d'amélioration continue, consultants.
    \item[Fonctionnalités Incluses :]
          \begin{itemize}
              \item Toutes les fonctionnalités de la version \textit{Community}.
              \item Nombre de projets illimité.
              \item Accès à tous les modes de collecte de données :
                    \begin{itemize}
                        \item \textbf{Mode Manuel} (avec l'interface de saisie opérateur).
                        \item \textbf{Mode Dynamique} (connexions aux bases de données SQL et API REST).
                    \end{itemize}
              \item Accès au dashboard de visualisation web.
              \item Suggestions d'optimisation intelligentes.
              \item Export sans filigrane.
              \item Support technique par email.
          \end{itemize}
    \item[Modèle de Prix Suggéré :] Abonnement mensuel ou annuel par utilisateur (ex: 49€/utilisateur/mois).
\end{description}

\subsubsection{Niveau 3 : VSM-Tools \textit{Enterprise} (Devis personnalisé)}
\begin{description}[leftmargin=*, style=unboxed]
    \item[Objectif :] Répondre aux besoins spécifiques des grandes organisations en matière de sécurité, d'intégration et de support.
    \item[Public Cible :] Grandes entreprises.
    \item[Fonctionnalités Incluses :]
          \begin{itemize}
              \item Toutes les fonctionnalités de la version \textit{Professional}.
              \item Authentification unique (Single Sign-On - SSO).
              \item Gestion avancée des utilisateurs et des droits d'accès (ACL).
              \item Support prioritaire avec garanties de temps de réponse (SLA).
              \item Option de déploiement sur une infrastructure dédiée (On-Premise ou Cloud privé).
              \item Accompagnement à la configuration et développement de connecteurs métiers spécifiques (en option).
          \end{itemize}
    \item[Modèle de Prix Suggéré :] Contrat annuel sur devis, basé sur le nombre d'utilisateurs et le niveau de service requis.
\end{description}

\subsection{Synthèse dans le Business Model Canvas}
L'ensemble de ces éléments est synthétisé dans le Business Model Canvas ci-dessous, qui offre une vue d'ensemble de la manière dont VSM-Tools crée, délivre et capture de la valeur.

\begin{figure}[H]
    \centering
    % \includegraphics[width=0.9\textwidth]{../../../images/business_model_canvas.png}
    \caption{Business Model Canvas pour VSM-Tools.}
    \label{fig:bmc}
\end{figure}

% --- SECTION 12 : ANALYSE DES COÛTS DU PROJET ---
\section{Analyse des coûts du projet}

Cette section détaille les coûts associés au développement du projet VSM-Tools. Bien que le projet soit mené dans un cadre académique où de nombreuses ressources sont fournies, cette analyse vise à simuler les investissements qu'un tel projet représenterait dans un contexte commercial, afin d'évaluer sa viabilité.

\subsection{Coûts de développement}
Les coûts de développement englobent les ressources humaines et matérielles nécessaires pour mener à bien la conception et la réalisation de l'application.

\subsubsection{Ressources humaines}
L'investissement principal pour ce projet réside dans le temps consacré par l'équipe de développement et l'encadrement.

L'équipe est composée de quatre élèves-ingénieurs qui ont dédié une part significative de leur cursus à ce projet. Le volume horaire total planifié, incluant les phases de conception, de développement, de tests et de préparation des livrables, est estimé à \textbf{1800 heures}. Dans un contexte commercial, ce volume de travail, valorisé à un taux horaire pour des ingénieurs juniors, représenterait le coût de R\&D principal.

Deux enseignants ont assuré l'encadrement du projet, apportant leur expertise et guidant l'équipe. Leur contribution, bien que non directement facturée, a une valeur considérable.

\subsubsection{Outils et logiciels}
Pour la réalisation de ce projet, l'équipe s'est appuyée sur un ensemble d'outils et de logiciels, majoritairement accessibles gratuitement ou via des programmes académiques :
\begin{itemize}[leftmargin=*]
    \item \textbf{Gestion de Projet :} Trello est utilisé pour le suivi des tâches (version gratuite).
    \item \textbf{Environnement de Développement :} Visual Studio Code (VSCode) est l'éditeur principal.
    \item \textbf{Contrôle de Version :} Git est utilisé pour le contrôle de version, avec des dépôts hébergés sur GitHub (plan gratuit).
    \item \textbf{Technologies :} L'écosystème logiciel (Electron, Node.js, React, PostgreSQL) est basé sur des technologies open source, n'induisant aucun coût de licence direct.
\end{itemize}
Dans un scénario d'entreprise, des coûts de licence pour des versions professionnelles ou des services cloud payants (ex: GitHub Enterprise) seraient à budgétiser.

\subsubsection{Infrastructure de développement}
Le développement a été effectué en local sur les postes des membres de l'équipe. L'hébergement du code et des serveurs de test a été géré via des offres gratuites ou des crédits étudiants (GitHub, plateformes cloud...). Par conséquent, les coûts d'infrastructure durant la phase de développement sont considérés comme nuls.

\subsection{Coûts opérationnels prévisionnels (post-développement)}
Une fois l'application VSM-Tools déployée et commercialisée, plusieurs catégories de coûts opérationnels récurrents devraient être anticipées pour assurer sa pérennité et sa croissance.

\begin{itemize}[leftmargin=*]
    \item \textbf{Hébergement :} Coûts mensuels ou annuels pour l'hébergement de la base de données PostgreSQL, du VSM Engine (Node.js) et des interfaces web sur une plateforme cloud (AWS, Azure, OVH...).
    \item \textbf{Maintenance Applicative :} Temps de développement continu pour la correction de bugs, les mises à jour de sécurité et le développement de nouvelles fonctionnalités.
    \item \textbf{Coûts d'API :} Coûts liés à l'utilisation de l'API du modèle de langage (IA conversationnelle), qui sont souvent basés sur le volume de requêtes.
    \item \textbf{Support Utilisateur :} Coûts associés aux outils de ticketing et potentiellement au personnel dédié pour assister les clients des offres payantes.
    \item \textbf{Marketing et Ventes :} Investissements pour acquérir de nouveaux clients (publicité en ligne, création de contenu, etc.).
    \item \textbf{Frais Administratifs et Généraux :} Coûts légaux, comptables, et autres frais de fonctionnement.
\end{itemize}

Le tableau ci-dessous synthétise une estimation de ces coûts opérationnels annuels, basée sur un lancement à petite échelle.

\begin{table}[H]
    \centering
    \caption{Détail des coûts opérationnels annuels prévisionnels post-lancement.}
    \label{tab:couts_operationnels_annuels}
    \begin{tabular}{|l|r|}
        \hline
        \textbf{Catégorie de Coût}                        & \textbf{Estimation Annuelle (€)} \\
        \hline
        Hébergement et Infrastructure Cloud               & 600 - 2 000                      \\
        \hline
        Maintenance Applicative (Corrective \& Évolutive) & 3 000 - 10 000                   \\
        \hline
        Coûts d'API (Intelligence Artificielle)           & 500 - 1 500                      \\
        \hline
        Support Utilisateur et Outils                     & 500 - 2 500                      \\
        \hline
        Marketing et Ventes                               & 2 000 - 8 000                    \\
        \hline
        Frais Administratifs et Généraux                  & 1 000 - 3 000                    \\
        \hline
        \textbf{Total Opérationnel Annuel}                & \textbf{7 600 - 27 000}          \\
        \hline
    \end{tabular}
\end{table}

Ces chiffres illustrent l'investissement potentiel que représenterait VSM-Tools s'il était opéré dans un cadre commercial, et soulignent la valeur significative apportée par le travail de l'équipe projet dans le contexte académique actuel.

% --- SECTION 13 : GESTION DE PROJET DE LA PHASE D'IMPLÉMENTATION ---
\section{Gestion de projet de la phase d'implémentation}

\subsection{Méthodologie utilisée}
Pour la phase de développement concrète du projet, l'approche \textbf{Agile}, inspirée de \textbf{Scrum}, a été maintenue et renforcée. Le choix de cette méthodologie s'est avéré particulièrement adapté à une phase d'implémentation, car elle offre :
\begin{itemize}[leftmargin=*]
    \item \textbf{Flexibilité :} La capacité de s'adapter aux défis techniques découverts en cours de codage.
    \item \textbf{Livraison Itérative :} Le travail a été découpé en \textbf{Sprints de deux semaines}. Chaque sprint avait pour objectif de produire un incrément fonctionnel et testable de l'application, permettant de voir des progrès concrets et réguliers.
    \item \textbf{Visibilité et Synchronisation :} Des réunions de suivi régulières ont permis à l'équipe de rester alignée, de partager les avancements et de résoudre rapidement les blocages.
\end{itemize}
Le suivi des tâches a été réalisé via un tableau visuel sur Trello, et le code source a été géré de manière collaborative avec Git et GitHub.

\subsection{Planification et répartition des tâches}
La phase d'implémentation a été planifiée sur une période de 11 semaines, du 30 Septembre au 11 Décembre 2025. Le travail a été structuré en cinq sprints de deux semaines, suivis d'une semaine de finalisation.

\begin{description}[leftmargin=*, style=unboxed]
    \item[Sprint 1 (Semaines 1-2) : Le Socle Technique.] L'objectif était de construire les fondations de l'application.
          \begin{itemize}
              \item Initialisation du projet Electron (VSM Studio) et du serveur Node.js (VSM Engine).
              \item Conception et création du schéma de la base de données PostgreSQL.
              \item Mise en place de la communication de base entre le Studio et l'Engine via une API REST interne.
          \end{itemize}

    \item[Sprint 2 (Semaines 3-4) : Le Cœur de la Modélisation.] L'objectif était d'implémenter la logique de création du modèle de données.
          \begin{itemize}
              \item Développement des premières sections du Dialogue de Configuration Central (Gestion des Sources de Données, des Nœuds Principaux).
              \item Implémentation des opérations CRUD (Create, Read, Update, Delete) dans l'Engine pour ces entités.
          \end{itemize}

    \item[Sprint 3 (Semaines 5-6) : Finalisation du Modèle et Rendu Visuel.] L'objectif était de pouvoir générer un premier diagramme.
          \begin{itemize}
              \item Implémentation des sections complexes du Dialogue (Séquençage du Flux, Indicateurs).
              \item Développement de la première version de l'algorithme de layout automatique pour le rendu sur le canevas.
          \end{itemize}

    \item[Sprint 4 (Semaines 7-8) : Dynamisation et Interfaces Externes.] L'objectif était de rendre la VSM "vivante".
          \begin{itemize}
              \item Implémentation des connecteurs de données (SQL, REST) dans l'Engine.
              \item Développement des interfaces web pour la saisie opérateur et le dashboard de visualisation.
              \item Intégration de l'agent conversationnel IA dans le Studio.
          \end{itemize}

    \item[Sprint 5 (Semaines 9-10) : Finalisation et Finitions.] L'objectif était de stabiliser le produit.
          \begin{itemize}
              \item Phase de tests approfondis et de correction de bugs.
              \item Amélioration de l'interface utilisateur et de l'expérience utilisateur (UI/UX).
              \item Implémentation des fonctionnalités d'import/export au format \texttt{.vsmx}.
          \end{itemize}

    \item[Semaine 11 : Clôture du Projet.]
          \begin{itemize}
              \item Préparation de la documentation finale.
              \item Répétition et finalisation de la soutenance du projet.
          \end{itemize}
\end{description}

\subsection{Diagramme de Gantt de la phase d'implémentation}
Le diagramme ci-dessous offre une représentation visuelle de la planification des sprints et des tâches clés sur la période du projet.

\begin{figure}[H]
    \centering
    \resizebox{\textwidth}{!}{
        \begin{ganttchart}[
                vgrid,
                hgrid,
                x unit=0.8cm,
                y unit title=0.5cm,
                y unit chart=0.5cm,
                title/.style={fill=blue!10},
                bar/.style={fill=blue!30},
                group/.style={fill=blue!20}
            ]{1}{11}
            \gantttitle{Planification de la phase d'implémentation (30/09 au 11/12/2025)}{11} \\
            \gantttitlelist{1,...,11}{1} \\

            % Sprint 1
            \ganttgroup{Sprint 1 : Socle Technique}{1}{2} \\
            \ganttbar{Mise en place des projets (Studio \& Engine)}{1}{1} \\
            \ganttbar{Schéma de la BDD PostgreSQL}{1}{2} \\
            \ganttbar{API interne de base}{2}{2} \\

            % Sprint 2
            \ganttgroup{Sprint 2 : Cœur de la Modélisation}{3}{4} \\
            \ganttbar{Dialogue : Sources de Données}{3}{3} \\
            \ganttbar{Dialogue : Nœuds Principaux}{3}{4} \\
            \ganttbar{Logique de persistance (CRUD)}{3}{4} \\

            % Sprint 3
            \ganttgroup{Sprint 3 : Rendu Visuel}{5}{6} \\
            \ganttbar{Dialogue : Séquençage \& Indicateurs}{5}{6} \\
            \ganttbar{Algorithme de Layout v1}{5}{6} \\

            % Sprint 4
            \ganttgroup{Sprint 4 : Dynamisation}{7}{8} \\
            \ganttbar{Connecteurs de données (SQL/REST)}{7}{7} \\
            \ganttbar{Interfaces Web (Opérateur/Dashboard)}{7}{8} \\
            \ganttbar{Intégration de l'IA}{8}{8} \\

            % Sprint 5
            \ganttgroup{Sprint 5 : Finitions}{9}{10} \\
            \ganttbar{Tests et débogage}{9}{10} \\
            \ganttbar{Amélioration UI/UX}{9}{10} \\
            \ganttbar{Import/Export .vsmx}{10}{10} \\

            % Semaine 11
            \ganttgroup{Sprint 6 : Clôture}{11}{11} \\
            \ganttbar{Rédaction rapport final}{11}{11} \\
            \ganttbar{Préparation soutenance}{11}{11}

        \end{ganttchart}
    }
    \caption{Diagramme de Gantt de la phase d'implémentation du projet.}
    \label{fig:gantt_impl}
\end{figure}

% --- SECTION 14 : CONCLUSION ---
\section{Conclusion}

Au terme de ce projet, la solution VSM-Tools se présente comme une réponse concrète et innovante aux défis de l'optimisation industrielle. Parti du constat que la visibilité des flux est le prérequis à toute amélioration, ce projet a abouti à la création d'un écosystème logiciel complet, conçu pour supporter le cycle intégral de la méthodologie VSM : du diagnostic de l'\textbf{état actuel} à la conception et la quantification de l'\textbf{état futur}.

La démarche de conception, centrée sur une approche \enquote{Model-First}, a permis de développer un outil qui guide l'utilisateur et garantit la cohérence des cartographies. En s'appuyant sur une architecture technique moderne et découplée (Electron, Node.js, React, PostgreSQL), nous avons construit une plateforme robuste, capable de s'adapter à différents niveaux de maturité digitale grâce à ses trois modes de collecte de données. L'intégration d'un agent conversationnel et de fonctionnalités d'analyse intelligentes positionne VSM-Tools au-delà d'un simple outil de dessin, en tant que véritable assistant à l'amélioration continue.

Ce projet a été bien plus qu'un simple exercice technique. Il a nécessité une réflexion approfondie sur les besoins des utilisateurs, sur la stratégie produit et sur la viabilité économique d'une telle solution, comme en témoigne le modèle économique proposé.

Bien que la version actuelle constitue un socle solide, le potentiel de VSM-Tools est encore vaste. Les perspectives d'évolution, notamment l'intégration de capacités de simulation pour valider l'état futur et le renforcement de la collaboration, tracent une feuille de route claire pour l'avenir.

En synthèse, VSM-Tools démontre qu'il est possible d'allier la rigueur d'une méthodologie Lean éprouvée à la puissance des technologies logicielles modernes, pour créer un outil à forte valeur ajoutée qui accompagne efficacement les entreprises, de l'analyse de leurs processus jusqu'à la conception et la mise en œuvre de leur excellence opérationnelle.

\newpage

% --- ANNEXES ---
\appendix
\section{Annexes}

\subsection{Annexe A : Exemple de VSM généré par VSM-Tools}
% Ici, vous mettriez une capture d'écran de votre application montrant un diagramme finalisé.
\begin{figure}[H]
    \centering
    \includegraphics[width=0.9\textwidth]{../../../images/imgfinals/}
    \caption{Exemple de VSM - État Actuel.}
    \label{fig:vsm_actuel}
\end{figure}

\subsection{Annexe B : Maquettes de l'Interface Utilisateur}
% Ici, vous pourriez mettre les captures d'écran des fenêtres de dialogue que vous m'avez montrées (création de source, édition d'indicateur).

\subsection{Annexe C : Business Model Canvas}
% L'image du Business Model Canvas que vous avez créé.

\vspace{1cm}
\subsection{Dépôt GitHub du Projet}
\noindent Le code source du projet est disponible sur GitHub : \url{https://github.com/daniozo/VSM-Tools}

% --- FIN DU DOCUMENT ---
\end{document}